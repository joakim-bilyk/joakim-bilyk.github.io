% Options for packages loaded elsewhere
\PassOptionsToPackage{unicode}{hyperref}
\PassOptionsToPackage{hyphens}{url}
%
\documentclass[
]{article}
\usepackage{amsmath,amssymb}
\usepackage{lmodern}
\usepackage{iftex}
\ifPDFTeX
  \usepackage[T1]{fontenc}
  \usepackage[utf8]{inputenc}
  \usepackage{textcomp} % provide euro and other symbols
\else % if luatex or xetex
  \usepackage{unicode-math}
  \defaultfontfeatures{Scale=MatchLowercase}
  \defaultfontfeatures[\rmfamily]{Ligatures=TeX,Scale=1}
\fi
% Use upquote if available, for straight quotes in verbatim environments
\IfFileExists{upquote.sty}{\usepackage{upquote}}{}
\IfFileExists{microtype.sty}{% use microtype if available
  \usepackage[]{microtype}
  \UseMicrotypeSet[protrusion]{basicmath} % disable protrusion for tt fonts
}{}
\makeatletter
\@ifundefined{KOMAClassName}{% if non-KOMA class
  \IfFileExists{parskip.sty}{%
    \usepackage{parskip}
  }{% else
    \setlength{\parindent}{0pt}
    \setlength{\parskip}{6pt plus 2pt minus 1pt}}
}{% if KOMA class
  \KOMAoptions{parskip=half}}
\makeatother
\usepackage{xcolor}
\usepackage[margin=1in]{geometry}
\usepackage{graphicx}
\makeatletter
\def\maxwidth{\ifdim\Gin@nat@width>\linewidth\linewidth\else\Gin@nat@width\fi}
\def\maxheight{\ifdim\Gin@nat@height>\textheight\textheight\else\Gin@nat@height\fi}
\makeatother
% Scale images if necessary, so that they will not overflow the page
% margins by default, and it is still possible to overwrite the defaults
% using explicit options in \includegraphics[width, height, ...]{}
\setkeys{Gin}{width=\maxwidth,height=\maxheight,keepaspectratio}
% Set default figure placement to htbp
\makeatletter
\def\fps@figure{htbp}
\makeatother
\setlength{\emergencystretch}{3em} % prevent overfull lines
\providecommand{\tightlist}{%
  \setlength{\itemsep}{0pt}\setlength{\parskip}{0pt}}
\setcounter{secnumdepth}{-\maxdimen} % remove section numbering
\usepackage{wrapfig}
\usepackage{graphics}
\usepackage{fancyhdr}
\pagenumbering{gobble}
\addtocontents{toc}{\protect\thispagestyle{empty}}
\addtocontents{title}{\protect\thispagestyle{empty}}
\ifLuaTeX
  \usepackage{selnolig}  % disable illegal ligatures
\fi
\IfFileExists{bookmark.sty}{\usepackage{bookmark}}{\usepackage{hyperref}}
\IfFileExists{xurl.sty}{\usepackage{xurl}}{} % add URL line breaks if available
\urlstyle{same} % disable monospaced font for URLs
\hypersetup{
  pdftitle={Exam-prep (FinKont)},
  pdfauthor={Joakim Bilyk},
  hidelinks,
  pdfcreator={LaTeX via pandoc}}

\title{Exam-prep (FinKont)}
\author{Joakim Bilyk}
\date{January 24, 2023\newpage}

\begin{document}
\maketitle

{
\setcounter{tocdepth}{2}
\tableofcontents
}
\thispagestyle{empty}
\newpage
\setcounter{page}{1}
\pagenumbering{arabic}
\pagestyle{fancy}

\hypertarget{exam-sets-201718-202122}{%
\section{Exam sets (2017/18-2021/22)}\label{exam-sets-201718-202122}}

\hypertarget{in-progress}{%
\subsection{In progress}\label{in-progress}}

\hypertarget{problem-3}{%
\subsubsection{Problem 3}\label{problem-3}}

Consider a two-dimensional model. The market model consist of three
assets: A bank account \(B_t\) and two stocks \(S_1\) and \(S_2\). The
\(P\)-dynamics of \(B_t\) is

\[
dB_t=rB_t\ dt,\ B_0=1,
\]

where \(r\in\mathbb{R}\) is a constant interest rate. The \(P\)-dynamics
of \(S_1\) and \(S_2\) are given by

\begin{align*}
dS_1(t)&=\alpha_1S_1(t)\ dt+\sigma_1S_1(t)\ d\overline{W}_1(t),&S_1(0)=s_1>0,\\
dS_2(t)&=\alpha_2S_2(t)\ dt+\sigma_2S_2(t)\ d\overline{W}_2(t),&S_2(0)=s_2>0,
\end{align*}

where \(\alpha_1,\alpha_2\in\mathbb{R}\) are constants. Moreover,
\(\sigma_1>0\) is a constant and \(\sigma_2(t)=\sigma_0e^{-\gamma t}\)
where \(\sigma_0>0\) and \(\gamma>0\) are constants and
\(\overline{W}_1(t)\) and \(\overline{W}_2(t)\) are two independent
\(P\)-Brownian motions. The filtration is the one generated by the two
Brownian motions, that is,
\(\mathcal{F}_t=\sigma(\overline{W}_1(s),\overline{W}_2(s)\ \vert\ 0\le s\le t)\).
Let \(T>0\) be a given and fixed (expiry) date.

\begin{enumerate}
\def\labelenumi{\alph{enumi}.}
\item
  \begin{enumerate}
  \def\labelenumii{\roman{enumii}.}
  \tightlist
  \item
    Is the model arbitrage free?
  \item
    Is the model complete?
  \end{enumerate}
\end{enumerate}

Consider the derivative that at time \(T\) pays \(X=S_1(T)S_2(T)\) and
let \(F(t,s_1,s_2)\) be the pricing function of the derivative.

\begin{enumerate}
\def\labelenumi{\alph{enumi}.}
\setcounter{enumi}{1}
\item
  \begin{enumerate}
  \def\labelenumii{\roman{enumii}.}
  \tightlist
  \item
    Determine the arbitrage free price of derivative \(X\) at time
    \(t=0\).
  \item
    Determine the equation satisfied by the pricing function
    \(F(t,s_1,s_2)\).
  \end{enumerate}
\end{enumerate}

Consider a new derivative that at time \(T\) pays \(Y=\log(S_2(T))\).

\begin{enumerate}
\def\labelenumi{\alph{enumi}.}
\setcounter{enumi}{2}
\tightlist
\item
  Determine the arbitrage free price of derivative \(Y\) at time
  \(t<T\).
\end{enumerate}

\textbf{Solution (a).}

\textbf{Solution (b).}

\textbf{Solution (c).}

\hypertarget{exam-201718}{%
\subsection{Exam 2017/18}\label{exam-201718}}

\hypertarget{problem-1}{%
\subsubsection{Problem 1}\label{problem-1}}

Let \(W_t\) denote a Brownian motion and let

\[
\mathcal{F}_t=\mathcal{F}_t^W=\sigma(\{W_s\ \vert\ 0\le s\le t\}).
\]

Let \(T>0\) be a given and fixed time.

Let \(f(t)\) be a bounded deterministic continuous function. Define the
two processes

\[
\begin{cases}
X_t=\int_0^tf(u)\ dW_u,\\
M^{(\lambda)}_t=\exp\left\{\lambda X_t-\frac{\lambda^2}{2}\int_0^t f^2(u)\ du\right\},
\end{cases}
\]

where \(\lambda\in\mathbb{R}\) is a constant.

\begin{enumerate}
\def\labelenumi{\alph{enumi}.}
\tightlist
\item
  Show that \(M^{(\lambda)}\) is a martingale with
  \(E[M_t^{(\lambda)}]=1\).
\end{enumerate}

Let \(0<s<t\) and \(\lambda_1,\lambda_2\in \mathbb{R}\) be given and
fixed.

\begin{enumerate}
\def\labelenumi{\alph{enumi}.}
\setcounter{enumi}{1}
\item
  \begin{enumerate}
  \def\labelenumii{\roman{enumii}.}
  \tightlist
  \item
    Show that
  \end{enumerate}

  \begin{align*}
  M^{(\lambda_1)}_s&=E\left[\left.\frac{M^{(\lambda_1)}_sM^{(\lambda_2)}_t}{M^{(\lambda_2)}_s} \ \right\vert\ \mathcal{F}_s\right]\\
  &=E\left[\left.\exp\left\{\lambda_1X_s+\lambda_2(X_t-X_s)-\frac{1}{2}\lambda_1^2\int_0^sf^2(u)\ du- \frac{1}{2}\lambda_2^2\int_s^tf^2(u)\ du\right\} \ \right\vert\ \mathcal{F}_s\right]
  \end{align*}

  \begin{enumerate}
  \def\labelenumii{\roman{enumii}.}
  \setcounter{enumii}{1}
  \tightlist
  \item
    Show that \(X_s\) and \(X_t-X_s\) are normally distributed and
    independent.
  \end{enumerate}
\item
  Compute the mean value of \(M^{(\lambda)}_T\log(M^{(\lambda)}_T)\).
\end{enumerate}

\textbf{Solution (a).}

First, we see that since \(X_t\) is on integral form we know that

\[
\begin{cases}
dX_t=f(t)\ dW_t\\
X_0=0.
\end{cases}
\]

Hence we may represent \(M\) as \(M^{(\lambda)}_t=g(t,X_t,Y_t)\) given
by

\[
g(t,x,y)=\exp\left\{\lambda x-\frac{\lambda^2}{2}y \right\},
\]

where \(Y_t=\int_0^t f^2(u)\ du\) with dynamics

\[
\begin{cases}
dY_t=f^2(t)\ dt\\
Y_0=0.
\end{cases}
\]

Hence by the multidimensional Ito's formula we have the dynamics of
\(M\) given by

\begin{align*}
dM^{(\lambda)}_t&=g_t\ dt+g_x\ dX_t+g_y\ dY_t+\frac{1}{2}g_{yy}\ (dY_t)^2+\frac{1}{2}g_{xx}\ (dX_t)^2 +f_{xy}(dX_t)(dY_t)\\
&=0+\lambda g\ dX_t-\frac{\lambda^2}{2}g\ dY_t+0+\frac{1}{2}\lambda ^2g\ (dX_t)^2+0\\
&=\lambda M_t^{(\lambda)} f(t)\ dW_t-\frac{1}{2}\lambda^2M_t^{(\lambda)} f^2(t)\ dt+\frac{1}{2}\lambda M_t^{(\lambda)} f^2(t)\ dt\\
&=\lambda f(t)M_t^{(\lambda)}\ dW_t,
\end{align*}

And so we see that \(M\) is a martingale as it only has dynamics wrt.
the Brownian motion \(W\) (assuming
\(\lambda f_tM_t^{(\lambda)}\in\mathcal{L}^2\)). Furthermore we have
that

\[
M_0^{(\lambda)}=g(0,X_0,Y_0)=\exp\left\{\lambda X_0-\frac{1}{2}\lambda ^2 Y_0\right\}=e^0=1
\]

and so we have \(E[M_t^{(\lambda)}]=M_0^{(\lambda)}=1\) as desired.
\(\square\)

\textbf{Solution (b).}

\emph{``(i)''} We have from the previous exercise

\begin{align*}
&\frac{M^{(\lambda_1)}_sM^{(\lambda_2)}_t}{M^{(\lambda_2)}_s}\\
&=\exp\left\{\lambda_1 X_s-\frac{1}{2}\lambda_1^2\int_0^s f^2(u)\ du\right\}\exp\left\{\lambda_2 X_t-\frac{1}{2}\lambda_2^2\int_0^t f^2(u)\ du\right\}\exp\left\{\frac{1}{2}\lambda_2^2\int_0^s f^2(u)\ du-\lambda_2 X_s\right\}\\
&=\exp\left\{\lambda_1 X_s-\frac{1}{2}\lambda_1^2\int_0^s f^2(u)\ du+\lambda_2 X_t-\frac{1}{2}\lambda_2^2\int_0^t f^2(u)\ du+\frac{1}{2}\lambda_2^2\int_0^s f^2(u)\ du-\lambda_2 X_s\right\}\\
&=\exp\left\{\lambda_1 X_s+\lambda_2 (X_t-X_s)-\frac{1}{2}\lambda_1^2\int_0^s f^2(u)\ du-\frac{1}{2}\lambda_2^2\int_s^t f^2(u)\ du\right\}
\end{align*}

and so the conclusion follows. \(\square\)

\emph{``(ii)''} We have that from lemma 4.18 that

\[
X_s=\int_0^sf(u)\ dW_u\sim \mathcal{N}\left(0,\int_0^sf^2(u)\ dW_u\right)
\]

furthermore we have that

\[
X_t-X_s=\int_s^tf(u)\ dW_u\sim \mathcal{N}\left(0,\int_s^tf^2(u)\ dW_u\right).
\]

In regard to the independence claim we could check identity below

\[
E[e^{t_1X}e^{t_2 Y}]=E[e^{t_1X}]E[e^{t_2Y}]
\]

where \(X,Y\) are independent random variables. The above identity holds
if and only if \(X\) and \(Y\) are independent. From above we have that

\[
M_s^{(\lambda_1)}=E[e^{\lambda_1X_s}e^{\lambda_2(X_t-X_s)}\ \vert\ \mathcal{F}_s]e^{-\frac{1}{2}\lambda_1^2\int_0^s f^2(u)\ du-\frac{1}{2}\lambda_2^2\int_s^t f^2(u)\ du}
\]

and so taking expectation we have

\[
1=E[e^{\lambda_1X_s}e^{\lambda_2(X_t-X_s)}]e^{-\frac{1}{2}\lambda_1^2\int_0^s f^2(u)\ du-\frac{1}{2}\lambda_2^2\int_s^t f^2(u)\ du}
\] Which the gives

\[
E[e^{\lambda_1X_s}e^{\lambda_2(X_t-X_s)}]=e^{\frac{1}{2}\lambda_1^2\int_0^s f^2(u)\ du+\frac{1}{2}\lambda_2^2\int_s^t f^2(u)\ du}=E[e^{\lambda_1X_s}]E[e^{\lambda_2(X_t-X_s)}]
\]

and so the conclusion is that \(X_s\) and \(X_t-X_s\) are independent.
\(\square\)

\textbf{Solution (c).}

We recall the definition of \(M_t^{(\lambda)}\) and observe that

\[
\log M_t^{(\lambda)}=\lambda X_t-\frac{1}{2}\lambda ^2\int_0^t f^2(u)\ du.
\]

Furthermore we have the dynamics of \(M^{(\lambda)}\) given by the
differential form

\[
dM_t^{(\lambda)}=\lambda f(t)M_t^{(\lambda)}\ dW_t.
\]

with \(M_0^{(\lambda)}=1\). Since we know that \(M_t^{(\lambda)}\) is a
martingale we have

\[
E^P[M_T^{(\lambda)}]=E^P[M_0^{(\lambda)}]=1,
\]

and so we may define a new probability measure as

\[
d\tilde{P}=M_T^{(\lambda)}\ dP
\]

on \(\mathcal{F}_T\). We then have a new Brownian motion \(\tilde{W}\)
such that

\[
dW_t=\lambda f(t)\ dt + d\tilde{W}_t.
\]

We can then see

\begin{align*}
E^P[M_T^{(\lambda)}\log M_T^{(\lambda)}]&=\int M_T^{(\lambda)}\log M_T^{(\lambda)}\ dP=\int M_T^{(\lambda)}\log M_T^{(\lambda)} \frac{1}{M_T^{(\lambda)}}\ d\tilde{P}\\
&=\int \log M_T^{(\lambda)}\ d\tilde{P}=E^{\tilde{P}}[\log M_T^{(\lambda)}].
\end{align*}

Then we can evaluate the mean value by seeing the \(X\) has
representation wrt. \(\tilde{P}\) by

\[
X_t=\int_0^tf(u)\ (\lambda f(u)\ du + d\tilde{W}_u)=\lambda\int_0^tf^2(u)\ du+\int_0^tf(u)\ d\tilde{W}_u.
\]

Giving that

\begin{align*}
E^P[M_T^{(\lambda)}\log M_T^{(\lambda)}]&=E^{\tilde{P}}[\log M_T^{(\lambda)}]\\
&=E^{\tilde{P}}\left[ \lambda X_T-\frac{1}{2}\lambda ^2\int_0^T f^2(u)\ du \right]\\
&=E^{\tilde{P}}\left[ \lambda^2\int_0^Tf^2(u)\ du+\lambda\int_0^Tf(u)\ d\tilde{W}_u-\frac{1}{2}\lambda ^2\int_0^T f^2(u)\ du \right]\\
&=\lambda E^{\tilde{P}}\left[\frac{1}{2} \lambda\int_0^Tf^2(u)\ du+\int_0^Tf(u)\ d\tilde{W}_u \right]\\
&=\frac{1}{2} \lambda^2\int_0^Tf^2(u)\ du+\lambda E^{\tilde{P}}\left[\int_0^Tf(u)\ d\tilde{W}_u \right]\\
&=\frac{1}{2} \lambda^2\int_0^Tf^2(u)\ du
\end{align*}

Since

\[
\tilde{X}_T=\int_0^Tf(u)\ d\tilde{W}_u,
\]

is a \(\tilde{P}\)-martingale. \(\square\)

\hypertarget{problem-2}{%
\subsubsection{Problem 2}\label{problem-2}}

Consider a standard Black-Scholes model, that is, a model consisting of
a bank account \(B_t\) with \(P\)-dynamics given by

\[
dB_t=rB_t\ dt,\ B_0=1
\]

and a stock \(S_t\) with \(P\)-dynamics given by

\[
dS_t=\alpha S_t\ dt+\sigma S_t\ d\overline{W}_t,\ S_0=s>0
\]

where \(r,\alpha\in\mathbb{R}\) and \(\sigma >0\) are constants and
\(\overline{W}_t\) is a \(P\)-Brownian motion. Let \(T>0\) be a given
and fixed date.

Consider the derivative that at time \(T\) pays \[
X=\max\left\{\min\left\{S_T,K_2\right\},K_1\right\},
\]

where \(0<K_1<K_2\) are constants.

\begin{enumerate}
\def\labelenumi{\alph{enumi}.}
\tightlist
\item
  Determine the arbitrage free price of derivative \(X\) at time
  \(t<T\).
\end{enumerate}

Consider a new derivative that at time \(T\) pays

\[
Y=(S^2_T-K^2)^+-(K^2-S^2_T)^+.
\]

\begin{enumerate}
\def\labelenumi{\alph{enumi}.}
\setcounter{enumi}{1}
\item
  \begin{enumerate}
  \def\labelenumii{\roman{enumii}.}
  \tightlist
  \item
    Determine the arbitrage free price of derivative \(Y\) at time
    \(t<T\).
  \item
    Find a hedging portfolio for derivative \(Y\).
  \end{enumerate}
\end{enumerate}

Let \(h(t)=(h_0(t),h_1(t))\) be a portfolio where

\[
h_0(t)=-e^{r(T-2t)+\sigma^2(T-t)}S^2(t)
\]

is the number of units in the bank account at time \(t\) and

\[
h_1(t)=2e^{(r+\sigma^2)(T-t)}S(t)
\]

is the number of shares in the stock at time \(t\). Let \(V^h(t)\)
denote the associated value process.

\begin{enumerate}
\def\labelenumi{\alph{enumi}.}
\setcounter{enumi}{2}
\tightlist
\item
  Determine whether the portfolio \(h\) is self-financing or not.
\item
  Compute \(V^h(T)\).
\end{enumerate}

\textbf{Solution (a).}

We see that the derivative is the bull spread given by the payout
function

\[
X=
\begin{cases}
  K_2 & \text{if }S_T>K_2,\\
  S_T & \text{if }K_1\le S_T\le K_2,\\
  K_1 &\text{if }S_T< K_1.
\end{cases}
\]

We know from exercise 10.3 that this can be replicated by holding
\(K_1\) bonds, one call option with strike \(K_1\) and a short on a call
with strike \(K_2\). The arbitrage free price of the derivative is then
the value process of the mentioned portfolio i.e.

\[
\Pi_t[X]=K_1 e^{-r(T-t)}+c(K_1;t,T)-c(K_2;t,T),
\]

where \(c\) denotes the pricing function for a European call option
(non-instructive parameters supressed). \(\square\)

\textbf{Solution (b).}

\emph{(i)}: We start by seeing that the derivative pays out

\[
Y=
\begin{cases}
  S_T^2-K^2 & \text{if }S_T^2\ge K^2,\\
  -(K^2-S_T^2) &\text{if }S_T^2< K^2.
\end{cases}
\]

hence the payout is \(Y=S_T^2-K^2=\Phi(S_T)\) where \(\Phi(s)=s^2-K^2\).
That is \(Y\) is in fact a simple claim. We have from the risk neutral
valueation formula 7.11 that

\begin{align*}
\Pi_t[Y]&=e^{-r(T-t)}E^Q_{t,s}[S_T^2-K^2]\\
&=e^{-r(T-t)}E^Q_{t,s}[S_T^2]-e^{-r(T-t)}K^2.
\end{align*}

Recall that under the martingale measure \(Q\) we have that \(S_t\) is a
GBM hence

\[
S_t=s\cdot \exp\left\{\left(r-\frac{1}{2}\sigma^2\right)(T-t)+\sigma\left(W_T^Q-W_t^Q\right)\right\}
\]

then

\[
S_T^2=s^2\cdot \exp\left\{2\left(r-\frac{1}{2}\sigma^2\right)(T-t)+2\sigma\left(W_T^Q-W_t^Q\right)\right\}.
\]

Inserting this into the risk neutral valuation formula we get

\begin{align*}
\Pi_t[Y]&=e^{-r(T-t)}E^Q_{t,s}[S_T^2]-e^{-r(T-t)}K^2\\
&=e^{-r(T-t)}s^2e^{2\left(r-\frac{1}{2}\sigma^2\right)(T-t)} E^Q\left[\exp\left\{2\sigma\left(W_T^Q-W_t^Q\right)\right\}\right]-e^{-r(T-t)}K^2\\
&=e^{-r(T-t)}s^2e^{2\left(r-\frac{1}{2}\sigma^2\right)(T-t)}e^{\frac{1}{2}4\sigma^2(T-t)}-e^{-r(T-t)}K^2\\
&=e^{-r(T-t)}\left(s^2e^{(2r-\sigma^2)(T-t)+\frac{1}{2}4\sigma^2(T-t)}-K^2\right)\\
&=e^{-r(T-t)}\left(s^2e^{(2r+\sigma^2)(T-t)}-K^2\right).
\end{align*}

The arbitrage free price of the derivative is then given above.
\(\square\)

\emph{(ii)}: From theorem 8.5 we can determine a hedging portfolio with
weightings

\begin{align*}
w_t^B&=\frac{\Pi_t-S_t\frac{\partial\Pi}{\partial s}}{\Pi_t}\\
&=1-\frac{S_t2S_te^{-r(T-t)}e^{(2r+\sigma^2)(T-t)}}{e^{-r(T-t)}\left(S_t^2e^{(2r+\sigma^2)(T-t)}-K^2\right)}\\
&=1-\frac{2S_t^2e^{(2r+\sigma^2)(T-t)}}{S_t^2e^{(2r+\sigma^2)(T-t)}-K^2}\\
&=1-\frac{2}{1-K^2S_t^{-2}e^{(2r+\sigma^2)(t-T)}}\\
w_t^S&=\frac{2}{1-K^2S_t^{-2}e^{(2r+\sigma^2)(t-T)}}.
\end{align*}

In absolute terms we will hold the portfolio

\begin{align*}
h_t^S&=2S_te^{-r(T-t)}e^{(2r+\sigma^2)(T-t)}\\
h_t^B&=\frac{e^{-r(T-t)}\left(s^2e^{(2r+\sigma^2)(T-t)}-K^2\right)-S_th_t^S}{B_t}\\
&=\frac{e^{-r(T-t)}\left(s^2e^{(2r+\sigma^2)(T-t)}-K^2\right)-S_th_t^S}{e^{rt}}\\
&=e^{-rT}s^2e^{(2r+\sigma^2)(T-t)}-e^{-rT}K^2-e^{-rt}S_th_t^S.
\end{align*}

The portfolio above will hedge \(Y\) with probability one. \(\square\)

\textbf{Solution (c).}

We assume no dividends and no consumption that is \(c_t=0\) and
\(dD_t^i=0\) for \(i=0,1\). Then the portfolio is self-financing if and
only if the value process has dynamics.

\[
h_0(t)\ dB_t+h_1(t)\ dS_t=0
\]

This is given in lemma 6.12.

\textbf{THE BELOW IS IN WORKS AND NOT CORRECT!}

Now we have that the value process is given by

\[
V_t^h=h_0(t)B_t+h_1(t)S_t.
\]

Using the representation \(V_t^h=f(h_0(t),B_t)+f(h_1(t),S_t)\) given by
\(f(x,y)=xy\) we have

\[
dV_t^h=df(h_0(t),B_t)+df(h_1(t),S_t).
\]

Using Ito's formula on each term we have

\begin{align*}
df(h_0(t),B_t)&=B_t\ dh_0(t)+h_0(t)\ dB_t+(dB_t)(dh_0(t)),\\
df(h_1(t),S_t)&=S_t\ dh_1(t)+h_1(t)\ dS_t+(dS_t)(dh_1(t)),\\
\end{align*}

since of cause \(f_{xx}=f_{yy}=0\). We can the determine the dynamics of
the portfolio by

\begin{align*}
dh_0(t)&=-(-2t-\sigma^2)S_t^2e^{r(T-2t)+\sigma^2(T-t)}\ dt\\
&-2S_te^{r(T-2t)+\sigma^2(T-t)}\ dS_t\\
&-\frac{1}{2}2e^{r(T-2t)+\sigma^2(T-t)}\ (dS_t)^2\\
&=(-2t-\sigma^2)h_0(t)\ dt+\frac{2}{S_t}h_0(t)\ (\mu S_t\ dt+\sigma S_t\ dW_t)+\frac{1}{S_t^2}h_0(t) \sigma^2S_t^2\ dt\\
&=(\mu-1)2h_0(t)\ dt+2\sigma h_0(t)\ dW_t
\end{align*}

and

\begin{align*}
dh_1(t)&=(-r-\sigma^2)2e^{(r+\sigma^2)(T-t)}S_t\ dt\\
&+2e^{(r+\sigma^2)(T-t)}\ dS_t+0\\
&=(-r-\sigma^2)h_1(t)\ dt+\frac{1}{S_t}h_1(t)(\mu S_t\ dt+\sigma S_t\ dW_t)\\
&=(-r-\sigma^2+\mu)h_1(t)\ dt+h_1(t)\sigma \ dW_t\\
\end{align*}

And so in total

\begin{align*}
dV_t^h(t)&=df(h_0(t),B_t)+df(h_1(t),S_t)\\
&=B_t\ dh_0(t)+h_0(t)\ dB_t+(dB_t)(dh_0(t))\\
&+S_t\ dh_1(t)+h_1(t)\ dS_t+(dS_t)(dh_1(t))\\
&=B_t\ ((\mu-1)2h_0(t)\ dt+2\sigma h_0(t)\ dW_t)+h_0(t)\ rB_t\ dt+0\\
&+S_t\ ((-r-\sigma^2+\mu)h_1(t)\ dt+h_1(t)\sigma \ dW_t)+h_1(t)\ (\mu S_t\ dt+\sigma S_t\ dW_t)+\sigma^2S_th_1(t)\ dt\\
&=\left[B_t(\mu-1)2h_0(t)+h_0(t)rB_t+S_t(-r-\sigma^2+\mu)h_1(t)+h_1\mu S_t+\sigma^2S_th_1(t)\right]\ dt\\
&+\left[B_t2\sigma h_0(t)+S_th_1\sigma+h_1\sigma S_t\right]\ dW_t\\
&=\left[(2\mu-2+r)B_th_0(t)+(-r+2\mu)S_th_1(t)\right]\ dt\\
&+\left[B_t h_0(t)+h_1 S_t\right]2\sigma\ dW_t\\
&=V_t^h2\mu\ dt+V_t^h\ dW_t
\end{align*}

\textbf{Solution (d).}

We compute \(V_T^h\) easily by inserting \(h_0\) and \(h_1\) below

\begin{align*}
V_T^h&=B_Th_0(T)+S_Th_1(T)\\
&=B_T\left(-e^{r(T-2T)+\sigma^2(T-T)}S_T^2\right)+S_T\left(2e^{(r+\sigma^2)(T-T)}S_T\right)\\
&=-S_T^2+2S_T^2=S_T^2.
\end{align*}

and so \(h\) hedge the payout \(\Phi(S_T)=S_T^2\). \(\square\)

\hypertarget{exam-201819}{%
\subsection{Exam 2018/19}\label{exam-201819}}

Some content

\hypertarget{exam-201920}{%
\subsection{Exam 2019/20}\label{exam-201920}}

Some content

\hypertarget{exam-202021}{%
\subsection{Exam 2020/21}\label{exam-202021}}

Some content

\hypertarget{exam-202122}{%
\subsection{Exam 2021/22}\label{exam-202122}}

Some content

\hypertarget{exam-202223}{%
\subsection{Exam 2022/23}\label{exam-202223}}

Some content

\end{document}
