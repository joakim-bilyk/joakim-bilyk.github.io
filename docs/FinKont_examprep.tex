% Options for packages loaded elsewhere
\PassOptionsToPackage{unicode}{hyperref}
\PassOptionsToPackage{hyphens}{url}
%
\documentclass[
]{article}
\usepackage{amsmath,amssymb}
\usepackage{lmodern}
\usepackage{iftex}
\ifPDFTeX
  \usepackage[T1]{fontenc}
  \usepackage[utf8]{inputenc}
  \usepackage{textcomp} % provide euro and other symbols
\else % if luatex or xetex
  \usepackage{unicode-math}
  \defaultfontfeatures{Scale=MatchLowercase}
  \defaultfontfeatures[\rmfamily]{Ligatures=TeX,Scale=1}
\fi
% Use upquote if available, for straight quotes in verbatim environments
\IfFileExists{upquote.sty}{\usepackage{upquote}}{}
\IfFileExists{microtype.sty}{% use microtype if available
  \usepackage[]{microtype}
  \UseMicrotypeSet[protrusion]{basicmath} % disable protrusion for tt fonts
}{}
\makeatletter
\@ifundefined{KOMAClassName}{% if non-KOMA class
  \IfFileExists{parskip.sty}{%
    \usepackage{parskip}
  }{% else
    \setlength{\parindent}{0pt}
    \setlength{\parskip}{6pt plus 2pt minus 1pt}}
}{% if KOMA class
  \KOMAoptions{parskip=half}}
\makeatother
\usepackage{xcolor}
\usepackage{graphicx}
\usepackage[margin=1in]{geometry}
\usepackage{wrapfig}
\usepackage{graphics}
\usepackage{titling}
\usepackage{fancyhdr}
\pagenumbering{gobble}
\addtocontents{toc}{\protect\thispagestyle{empty}}
\addtocontents{title}{\protect\thispagestyle{empty}}
\setlength{\emergencystretch}{3em} % prevent overfull lines
\providecommand{\tightlist}{%
  \setlength{\itemsep}{0pt}\setlength{\parskip}{0pt}}
\setcounter{secnumdepth}{-\maxdimen} % remove section numbering
\ifLuaTeX
  \usepackage{selnolig}  % disable illegal ligatures
\fi
\IfFileExists{bookmark.sty}{\usepackage{bookmark}}{\usepackage{hyperref}}
\IfFileExists{xurl.sty}{\usepackage{xurl}}{} % add URL line breaks if available
\urlstyle{same} % disable monospaced font for URLs
\hypersetup{
  pdftitle={Exam Papers},
  pdfauthor={Joakim Bilyk},
  hidelinks,
  pdfcreator={LaTeX via pandoc}}

\title{Exam Papers}
\usepackage{etoolbox}
\makeatletter
\providecommand{\subtitle}[1]{% add subtitle to \maketitle
  \apptocmd{\@title}{\par {\large #1 \par}}{}{}
}
\makeatother
\subtitle{Continuous time finance (FinKont)}
\author{Joakim Bilyk}
\date{January 26, 2023}

\begin{document}
%\maketitle

{
%title
\begin{titlepage}
\newcommand{\HRule}{\rule{\linewidth}{0.5mm}} % Defines a new command for the horizontal lines, change thickness here

\center % Center everything on the page
 
%----------------------------------------------------------------------------------------
%	HEADING SECTIONS
%----------------------------------------------------------------------------------------

\textsc{\LARGE University of Copenhagen}\\[4cm] % Name of your university/college
\textsc{\Large Continuous time finance
(FinKont)}\\[0.5cm] % Major heading such as course name
%\textsc{\large Assignment 1}\\[0.5cm] % Minor heading such as course title

%----------------------------------------------------------------------------------------
%	TITLE SECTION
%----------------------------------------------------------------------------------------

\HRule \\[0.4cm]
{ \huge \bfseries \thetitle}\\[0.4cm] % Title of your document
\HRule \\[1.5cm]
 
%----------------------------------------------------------------------------------------
%	AUTHOR SECTION
%----------------------------------------------------------------------------------------

\begin{minipage}{0.4\textwidth}
\begin{flushleft} \large
\emph{Author:}\\
\textsc{\theauthor} \\
\end{flushleft}
\end{minipage}
~
\begin{minipage}{0.4\textwidth}
\begin{flushright} \large
\emph{Date:} \\
\textsc{\thedate} \\
\end{flushright}
\end{minipage}\\[2cm]

% If you don't want a supervisor, uncomment the two lines below and remove the section above
%\Large \emph{Author:}\\
%John \textsc{Smith}\\[3cm] % Your name

%----------------------------------------------------------------------------------------
%	DATE SECTION
%----------------------------------------------------------------------------------------

%{\large \thedate}\\[2cm] % Date, change the \today to a set date if you want to be precise

%----------------------------------------------------------------------------------------
%	LOGO SECTION
%----------------------------------------------------------------------------------------

\includegraphics[height=200px, keepaspectratio]{logo_ku.png}\\[4cm] % Include a department/university logo - this will require the graphicx package
 
%----------------------------------------------------------------------------------------

\vfill % Fill the rest of the page with whitespace
\end{titlepage}
\setcounter{tocdepth}{2}
\tableofcontents
}
\thispagestyle{empty}
\newpage
\setcounter{page}{1}
\pagenumbering{arabic}
\pagestyle{fancy}
\hypertarget{exam-sets}{%
\section{Exam sets}\label{exam-sets}}

\hypertarget{in-progress}{%
\subsection{In progress}\label{in-progress}}

\hypertarget{problem-2}{%
\subsubsection{Problem 2}\label{problem-2}}

Consider a standard Black-Scholes model, that is, a model consisting of
a bank account \(B(t)\) with \(P\)-dynamics given by

\[
dB(t)=rB(t)\ dt,
\]

with \(B(0)=1\) and a stock \(S(t)\) with \(P\)-dynamics given by

\[
dS(t)=\alpha S(t)\ dt+\sigma S(t)\ d\overline{W}(t),
\]

with \(S(0)=s>0\) where \(r,\alpha\in\mathbb{R}\) and \(\sigma >0\) are
constants and \(\overline{W}(t)\) is a \(P\)-Brownian motion. Let
\(T>0\) be a given and fixed (expiry) date.

Consider the derivative that at time \(T\) pays
\(X=\min\Big[\max\Big[S(T),K_1\Big],K_2\Big]\) where \(0<K_1<K_2\) are
constants. Let \(F(t,s)\) be the pricing function of the derivative.

\begin{enumerate}
\def\labelenumi{\alph{enumi}.}
\item
  \begin{enumerate}
  \def\labelenumii{\roman{enumii}.}
  \tightlist
  \item
    Determine the equations satisfied by the pricing function
    \(F(t,s)\).
  \item
    Find a hedging portfolio for the derivative \(X\). (Hint: Draw a
    picture of the payoff function).
  \end{enumerate}
\end{enumerate}

Let \(h(t)=\Big(h_0(t),h_1(t)\Big)\) be a self-financing portfolio given
by

\[
h_0(t)=(1-u)\frac{V^h(t)}{B(t)},\ h_1(t)=u\frac{V^h(t)}{S(t)}
\]

where \(u\) is a constant and set \(V^h(0)=1\). Note that \(h_0(t)\) is
the number of units of the bank account at time \(t\), and \(h_1(t)\) is
the number of shares in the stock at time \(t\), and \(V^h(t)\) denotes
the associated value process. Consider the derivative that at time \(T\)
pays \(Y=\sqrt{V^h(T)}\).

\begin{enumerate}
\def\labelenumi{\alph{enumi}.}
\setcounter{enumi}{1}
\tightlist
\item
  Determine the arbitrage free price of derivative \(Y\) at time
  \(t=0\).
\end{enumerate}

\textbf{Solution (a).}

\textbf{Solution (b).}

\hypertarget{problem-3}{%
\subsubsection{Problem 3}\label{problem-3}}

\textbf{Solution (a).}

\textbf{Solution (b).}

\textbf{Solution (c).}

\textbf{Solution (d).}

\textbf{Solution (e).}

\hypertarget{exam-201718}{%
\subsection{Exam 2017/18}\label{exam-201718}}

\hypertarget{problem-1}{%
\subsubsection{Problem 1}\label{problem-1}}

Let \(W_t\) denote a Brownian motion and let

\[
\mathcal{F}_t=\mathcal{F}_t^W=\sigma(\{W_s\ \vert\ 0\le s\le t\}).
\]

Let \(T>0\) be a given and fixed time.

Let \(f(t)\) be a bounded deterministic continuous function. Define the
two processes

\[
\begin{cases}
X_t=\int_0^tf(u)\ dW_u,\\
M^{(\lambda)}_t=\exp\left\{\lambda X_t-\frac{\lambda^2}{2}\int_0^t f^2(u)\ du\right\},
\end{cases}
\]

where \(\lambda\in\mathbb{R}\) is a constant.

\begin{enumerate}
\def\labelenumi{\alph{enumi}.}
\tightlist
\item
  Show that \(M^{(\lambda)}\) is a martingale with
  \(E[M_t^{(\lambda)}]=1\).
\end{enumerate}

Let \(0<s<t\) and \(\lambda_1,\lambda_2\in \mathbb{R}\) be given and
fixed.

\begin{enumerate}
\def\labelenumi{\alph{enumi}.}
\setcounter{enumi}{1}
\item
  \begin{enumerate}
  \def\labelenumii{\roman{enumii}.}
  \tightlist
  \item
    Show that
  \end{enumerate}

  \begin{align*}
  M^{(\lambda_1)}_s&=E\left[\left.\frac{M^{(\lambda_1)}_sM^{(\lambda_2)}_t}{M^{(\lambda_2)}_s} \ \right\vert\ \mathcal{F}_s\right]\\
  &=E\left[\left.\exp\left\{\lambda_1X_s+\lambda_2(X_t-X_s)-\frac{1}{2}\lambda_1^2\int_0^sf^2(u)\ du- \frac{1}{2}\lambda_2^2\int_s^tf^2(u)\ du\right\} \ \right\vert\ \mathcal{F}_s\right]
  \end{align*}

  \begin{enumerate}
  \def\labelenumii{\roman{enumii}.}
  \setcounter{enumii}{1}
  \tightlist
  \item
    Show that \(X_s\) and \(X_t-X_s\) are normally distributed and
    independent.
  \end{enumerate}
\item
  Compute the mean value of \(M^{(\lambda)}_T\log(M^{(\lambda)}_T)\).
\end{enumerate}

\textbf{Solution (a).}

First, we see that since \(X_t\) is on integral form we know that

\[
\begin{cases}
dX_t=f(t)\ dW_t\\
X_0=0.
\end{cases}
\]

Hence we may represent \(M\) as \(M^{(\lambda)}_t=g(t,X_t,Y_t)\) given
by

\[
g(t,x,y)=\exp\left\{\lambda x-\frac{\lambda^2}{2}y \right\},
\]

where \(Y_t=\int_0^t f^2(u)\ du\) with dynamics

\[
\begin{cases}
dY_t=f^2(t)\ dt\\
Y_0=0.
\end{cases}
\]

Hence by the multidimensional Ito's formula we have the dynamics of
\(M\) given by

\begin{align*}
dM^{(\lambda)}_t&=g_t\ dt+g_x\ dX_t+g_y\ dY_t+\frac{1}{2}g_{yy}\ (dY_t)^2+\frac{1}{2}g_{xx}\ (dX_t)^2 +f_{xy}(dX_t)(dY_t)\\
&=0+\lambda g\ dX_t-\frac{\lambda^2}{2}g\ dY_t+0+\frac{1}{2}\lambda ^2g\ (dX_t)^2+0\\
&=\lambda M_t^{(\lambda)} f(t)\ dW_t-\frac{1}{2}\lambda^2M_t^{(\lambda)} f^2(t)\ dt+\frac{1}{2}\lambda M_t^{(\lambda)} f^2(t)\ dt\\
&=\lambda f(t)M_t^{(\lambda)}\ dW_t,
\end{align*}

And so we see that \(M\) is a martingale as it only has dynamics wrt.
the Brownian motion \(W\) (assuming
\(\lambda f_tM_t^{(\lambda)}\in\mathcal{L}^2\)). Furthermore we have
that

\[
M_0^{(\lambda)}=g(0,X_0,Y_0)=\exp\left\{\lambda X_0-\frac{1}{2}\lambda ^2 Y_0\right\}=e^0=1
\]

and so we have \(E[M_t^{(\lambda)}]=M_0^{(\lambda)}=1\) as desired.
\(\square\)

\textbf{Solution (b).}

\emph{``(i)''} We have from the previous exercise

\begin{align*}
&\frac{M^{(\lambda_1)}_sM^{(\lambda_2)}_t}{M^{(\lambda_2)}_s}\\
&=\exp\left\{\lambda_1 X_s-\frac{1}{2}\lambda_1^2\int_0^s f^2(u)\ du\right\}\exp\left\{\lambda_2 X_t-\frac{1}{2}\lambda_2^2\int_0^t f^2(u)\ du\right\}\exp\left\{\frac{1}{2}\lambda_2^2\int_0^s f^2(u)\ du-\lambda_2 X_s\right\}\\
&=\exp\left\{\lambda_1 X_s-\frac{1}{2}\lambda_1^2\int_0^s f^2(u)\ du+\lambda_2 X_t-\frac{1}{2}\lambda_2^2\int_0^t f^2(u)\ du+\frac{1}{2}\lambda_2^2\int_0^s f^2(u)\ du-\lambda_2 X_s\right\}\\
&=\exp\left\{\lambda_1 X_s+\lambda_2 (X_t-X_s)-\frac{1}{2}\lambda_1^2\int_0^s f^2(u)\ du-\frac{1}{2}\lambda_2^2\int_s^t f^2(u)\ du\right\}
\end{align*}

and so the conclusion follows. \(\square\)

\emph{``(ii)''} We have that from lemma 4.18 that

\[
X_s=\int_0^sf(u)\ dW_u\sim \mathcal{N}\left(0,\int_0^sf^2(u)\ dW_u\right)
\]

furthermore we have that

\[
X_t-X_s=\int_s^tf(u)\ dW_u\sim \mathcal{N}\left(0,\int_s^tf^2(u)\ dW_u\right).
\]

In regard to the independence claim we could check identity below

\[
E[e^{t_1X}e^{t_2 Y}]=E[e^{t_1X}]E[e^{t_2Y}]
\]

where \(X,Y\) are independent random variables. The above identity holds
if and only if \(X\) and \(Y\) are independent. From above we have that

\[
M_s^{(\lambda_1)}=E[e^{\lambda_1X_s}e^{\lambda_2(X_t-X_s)}\ \vert\ \mathcal{F}_s]e^{-\frac{1}{2}\lambda_1^2\int_0^s f^2(u)\ du-\frac{1}{2}\lambda_2^2\int_s^t f^2(u)\ du}
\]

and so taking expectation we have

\[
1=E[e^{\lambda_1X_s}e^{\lambda_2(X_t-X_s)}]e^{-\frac{1}{2}\lambda_1^2\int_0^s f^2(u)\ du-\frac{1}{2}\lambda_2^2\int_s^t f^2(u)\ du}
\] Which the gives

\[
E[e^{\lambda_1X_s}e^{\lambda_2(X_t-X_s)}]=e^{\frac{1}{2}\lambda_1^2\int_0^s f^2(u)\ du+\frac{1}{2}\lambda_2^2\int_s^t f^2(u)\ du}=E[e^{\lambda_1X_s}]E[e^{\lambda_2(X_t-X_s)}]
\]

and so the conclusion is that \(X_s\) and \(X_t-X_s\) are independent.
\(\square\)

\textbf{Solution (c).}

We recall the definition of \(M_t^{(\lambda)}\) and observe that

\[
\log M_t^{(\lambda)}=\lambda X_t-\frac{1}{2}\lambda ^2\int_0^t f^2(u)\ du.
\]

Furthermore we have the dynamics of \(M^{(\lambda)}\) given by the
differential form

\[
dM_t^{(\lambda)}=\lambda f(t)M_t^{(\lambda)}\ dW_t.
\]

with \(M_0^{(\lambda)}=1\). Since we know that \(M_t^{(\lambda)}\) is a
martingale we have

\[
E^P[M_T^{(\lambda)}]=E^P[M_0^{(\lambda)}]=1,
\]

and so we may define a new probability measure as

\[
d\tilde{P}=M_T^{(\lambda)}\ dP
\]

on \(\mathcal{F}_T\). We then have a new Brownian motion \(\tilde{W}\)
such that

\[
dW_t=\lambda f(t)\ dt + d\tilde{W}_t.
\]

We can then see

\begin{align*}
E^P[M_T^{(\lambda)}\log M_T^{(\lambda)}]&=\int M_T^{(\lambda)}\log M_T^{(\lambda)}\ dP=\int M_T^{(\lambda)}\log M_T^{(\lambda)} \frac{1}{M_T^{(\lambda)}}\ d\tilde{P}\\
&=\int \log M_T^{(\lambda)}\ d\tilde{P}=E^{\tilde{P}}[\log M_T^{(\lambda)}].
\end{align*}

Then we can evaluate the mean value by seeing the \(X\) has
representation wrt. \(\tilde{P}\) by

\[
X_t=\int_0^tf(u)\ (\lambda f(u)\ du + d\tilde{W}_u)=\lambda\int_0^tf^2(u)\ du+\int_0^tf(u)\ d\tilde{W}_u.
\]

Giving that

\begin{align*}
E^P[M_T^{(\lambda)}\log M_T^{(\lambda)}]&=E^{\tilde{P}}[\log M_T^{(\lambda)}]\\
&=E^{\tilde{P}}\left[ \lambda X_T-\frac{1}{2}\lambda ^2\int_0^T f^2(u)\ du \right]\\
&=E^{\tilde{P}}\left[ \lambda^2\int_0^Tf^2(u)\ du+\lambda\int_0^Tf(u)\ d\tilde{W}_u-\frac{1}{2}\lambda ^2\int_0^T f^2(u)\ du \right]\\
&=\lambda E^{\tilde{P}}\left[\frac{1}{2} \lambda\int_0^Tf^2(u)\ du+\int_0^Tf(u)\ d\tilde{W}_u \right]\\
&=\frac{1}{2} \lambda^2\int_0^Tf^2(u)\ du+\lambda E^{\tilde{P}}\left[\int_0^Tf(u)\ d\tilde{W}_u \right]\\
&=\frac{1}{2} \lambda^2\int_0^Tf^2(u)\ du
\end{align*}

Since

\[
\tilde{X}_T=\int_0^Tf(u)\ d\tilde{W}_u,
\]

is a \(\tilde{P}\)-martingale. \(\square\)

\hypertarget{problem-2-1}{%
\subsubsection{Problem 2}\label{problem-2-1}}

Consider a standard Black-Scholes model, that is, a model consisting of
a bank account \(B_t\) with \(P\)-dynamics given by

\[
dB_t=rB_t\ dt,\ B_0=1
\]

and a stock \(S_t\) with \(P\)-dynamics given by

\[
dS_t=\alpha S_t\ dt+\sigma S_t\ d\overline{W}_t,\ S_0=s>0
\]

where \(r,\alpha\in\mathbb{R}\) and \(\sigma >0\) are constants and
\(\overline{W}_t\) is a \(P\)-Brownian motion. Let \(T>0\) be a given
and fixed date.

Consider the derivative that at time \(T\) pays \[
X=\max\left\{\min\left\{S_T,K_2\right\},K_1\right\},
\]

where \(0<K_1<K_2\) are constants.

\begin{enumerate}
\def\labelenumi{\alph{enumi}.}
\tightlist
\item
  Determine the arbitrage free price of derivative \(X\) at time
  \(t<T\).
\end{enumerate}

Consider a new derivative that at time \(T\) pays

\[
Y=(S^2_T-K^2)^+-(K^2-S^2_T)^+.
\]

\begin{enumerate}
\def\labelenumi{\alph{enumi}.}
\setcounter{enumi}{1}
\item
  \begin{enumerate}
  \def\labelenumii{\roman{enumii}.}
  \tightlist
  \item
    Determine the arbitrage free price of derivative \(Y\) at time
    \(t<T\).
  \item
    Find a hedging portfolio for derivative \(Y\).
  \end{enumerate}
\end{enumerate}

Let \(h(t)=(h_0(t),h_1(t))\) be a portfolio where

\[
h_0(t)=-e^{r(T-2t)+\sigma^2(T-t)}S^2(t)
\]

is the number of units in the bank account at time \(t\) and

\[
h_1(t)=2e^{(r+\sigma^2)(T-t)}S(t)
\]

is the number of shares in the stock at time \(t\). Let \(V^h(t)\)
denote the associated value process.

\begin{enumerate}
\def\labelenumi{\alph{enumi}.}
\setcounter{enumi}{2}
\tightlist
\item
  Determine whether the portfolio \(h\) is self-financing or not.
\item
  Compute \(V^h(T)\).
\end{enumerate}

\textbf{Solution (a).}

We see that the derivative is the bull spread given by the payout
function

\[
X=
\begin{cases}
  K_2 & \text{if }S_T>K_2,\\
  S_T & \text{if }K_1\le S_T\le K_2,\\
  K_1 &\text{if }S_T< K_1.
\end{cases}
\]

We know from exercise 10.3 that this can be replicated by holding
\(K_1\) bonds, one call option with strike \(K_1\) and a short on a call
with strike \(K_2\). The arbitrage free price of the derivative is then
the value process of the mentioned portfolio i.e.

\[
\Pi_t[X]=K_1 e^{-r(T-t)}+c(K_1;t,T)-c(K_2;t,T),
\]

where \(c\) denotes the pricing function for a European call option
(non-instructive parameters supressed). \(\square\)

\textbf{Solution (b).}

\emph{(i)}: We start by seeing that the derivative pays out

\[
Y=
\begin{cases}
  S_T^2-K^2 & \text{if }S_T^2\ge K^2,\\
  -(K^2-S_T^2) &\text{if }S_T^2< K^2.
\end{cases}
\]

hence the payout is \(Y=S_T^2-K^2=\Phi(S_T)\) where \(\Phi(s)=s^2-K^2\).
That is \(Y\) is in fact a simple claim. We have from the risk neutral
valueation formula 7.11 that

\begin{align*}
\Pi_t[Y]&=e^{-r(T-t)}E^Q_{t,s}[S_T^2-K^2]\\
&=e^{-r(T-t)}E^Q_{t,s}[S_T^2]-e^{-r(T-t)}K^2.
\end{align*}

Recall that under the martingale measure \(Q\) we have that \(S_t\) is a
GBM hence

\[
S_t=s\cdot \exp\left\{\left(r-\frac{1}{2}\sigma^2\right)(T-t)+\sigma\left(W_T^Q-W_t^Q\right)\right\}
\]

then

\[
S_T^2=s^2\cdot \exp\left\{2\left(r-\frac{1}{2}\sigma^2\right)(T-t)+2\sigma\left(W_T^Q-W_t^Q\right)\right\}.
\]

Inserting this into the risk neutral valuation formula we get

\begin{align*}
\Pi_t[Y]&=e^{-r(T-t)}E^Q_{t,s}[S_T^2]-e^{-r(T-t)}K^2\\
&=e^{-r(T-t)}s^2e^{2\left(r-\frac{1}{2}\sigma^2\right)(T-t)} E^Q\left[\exp\left\{2\sigma\left(W_T^Q-W_t^Q\right)\right\}\right]-e^{-r(T-t)}K^2\\
&=e^{-r(T-t)}s^2e^{2\left(r-\frac{1}{2}\sigma^2\right)(T-t)}e^{\frac{1}{2}4\sigma^2(T-t)}-e^{-r(T-t)}K^2\\
&=e^{-r(T-t)}\left(s^2e^{(2r-\sigma^2)(T-t)+\frac{1}{2}4\sigma^2(T-t)}-K^2\right)\\
&=e^{-r(T-t)}\left(s^2e^{(2r+\sigma^2)(T-t)}-K^2\right).
\end{align*}

The arbitrage free price of the derivative is then given above.
\(\square\)

\emph{(ii)}: From theorem 8.5 we can determine a hedging portfolio with
weightings

\begin{align*}
w_t^B&=\frac{\Pi_t-S_t\frac{\partial\Pi}{\partial s}}{\Pi_t}\\
&=1-\frac{S_t2S_te^{-r(T-t)}e^{(2r+\sigma^2)(T-t)}}{e^{-r(T-t)}\left(S_t^2e^{(2r+\sigma^2)(T-t)}-K^2\right)}\\
&=1-\frac{2S_t^2e^{(2r+\sigma^2)(T-t)}}{S_t^2e^{(2r+\sigma^2)(T-t)}-K^2}\\
&=1-\frac{2}{1-K^2S_t^{-2}e^{(2r+\sigma^2)(t-T)}}\\
w_t^S&=\frac{2}{1-K^2S_t^{-2}e^{(2r+\sigma^2)(t-T)}}.
\end{align*}

In absolute terms we will hold the portfolio

\begin{align*}
h_t^S&=2S_te^{-r(T-t)}e^{(2r+\sigma^2)(T-t)}\\
h_t^B&=\frac{e^{-r(T-t)}\left(s^2e^{(2r+\sigma^2)(T-t)}-K^2\right)-S_th_t^S}{B_t}\\
&=\frac{e^{-r(T-t)}\left(s^2e^{(2r+\sigma^2)(T-t)}-K^2\right)-S_th_t^S}{e^{rt}}\\
&=e^{-rT}s^2e^{(2r+\sigma^2)(T-t)}-e^{-rT}K^2-e^{-rt}S_th_t^S.
\end{align*}

The portfolio above will hedge \(Y\) with probability one. \(\square\)

\textbf{Solution (c).}

We assume no dividends and no consumption that is \(c_t=0\) and
\(dD_t^i=0\) for \(i=0,1\). Then the portfolio is self-financing if and
only if the value process has dynamics.

\[
h_0(t)\ dB_t+h_1(t)\ dS_t=0
\]

This is given in lemma 6.12.

\textbf{THE BELOW IS IN WORKS AND NOT CORRECT!}

Now we have that the value process is given by

\[
V_t^h=h_0(t)B_t+h_1(t)S_t.
\]

Using the representation \(V_t^h=f(h_0(t),B_t)+f(h_1(t),S_t)\) given by
\(f(x,y)=xy\) we have

\[
dV_t^h=df(h_0(t),B_t)+df(h_1(t),S_t).
\]

Using Ito's formula on each term we have

\begin{align*}
df(h_0(t),B_t)&=B_t\ dh_0(t)+h_0(t)\ dB_t+(dB_t)(dh_0(t)),\\
df(h_1(t),S_t)&=S_t\ dh_1(t)+h_1(t)\ dS_t+(dS_t)(dh_1(t)),\\
\end{align*}

since of cause \(f_{xx}=f_{yy}=0\). We can the determine the dynamics of
the portfolio by

\begin{align*}
dh_0(t)&=-(-2t-\sigma^2)S_t^2e^{r(T-2t)+\sigma^2(T-t)}\ dt\\
&-2S_te^{r(T-2t)+\sigma^2(T-t)}\ dS_t\\
&-\frac{1}{2}2e^{r(T-2t)+\sigma^2(T-t)}\ (dS_t)^2\\
&=(-2t-\sigma^2)h_0(t)\ dt+\frac{2}{S_t}h_0(t)\ (\mu S_t\ dt+\sigma S_t\ dW_t)+\frac{1}{S_t^2}h_0(t) \sigma^2S_t^2\ dt\\
&=(\mu-1)2h_0(t)\ dt+2\sigma h_0(t)\ dW_t
\end{align*}

and

\begin{align*}
dh_1(t)&=(-r-\sigma^2)2e^{(r+\sigma^2)(T-t)}S_t\ dt\\
&+2e^{(r+\sigma^2)(T-t)}\ dS_t+0\\
&=(-r-\sigma^2)h_1(t)\ dt+\frac{1}{S_t}h_1(t)(\mu S_t\ dt+\sigma S_t\ dW_t)\\
&=(-r-\sigma^2+\mu)h_1(t)\ dt+h_1(t)\sigma \ dW_t\\
\end{align*}

And so in total

\begin{align*}
dV_t^h(t)&=df(h_0(t),B_t)+df(h_1(t),S_t)\\
&=B_t\ dh_0(t)+h_0(t)\ dB_t+(dB_t)(dh_0(t))\\
&+S_t\ dh_1(t)+h_1(t)\ dS_t+(dS_t)(dh_1(t))\\
&=B_t\ ((\mu-1)2h_0(t)\ dt+2\sigma h_0(t)\ dW_t)+h_0(t)\ rB_t\ dt+0\\
&+S_t\ ((-r-\sigma^2+\mu)h_1(t)\ dt+h_1(t)\sigma \ dW_t)+h_1(t)\ (\mu S_t\ dt+\sigma S_t\ dW_t)+\sigma^2S_th_1(t)\ dt\\
&=\left[B_t(\mu-1)2h_0(t)+h_0(t)rB_t+S_t(-r-\sigma^2+\mu)h_1(t)+h_1\mu S_t+\sigma^2S_th_1(t)\right]\ dt\\
&+\left[B_t2\sigma h_0(t)+S_th_1\sigma+h_1\sigma S_t\right]\ dW_t\\
&=\left[(2\mu-2+r)B_th_0(t)+(-r+2\mu)S_th_1(t)\right]\ dt\\
&+\left[B_t h_0(t)+h_1 S_t\right]2\sigma\ dW_t\\
&=V_t^h2\mu\ dt+V_t^h\ dW_t
\end{align*}

\textbf{Solution (d).}

We compute \(V_T^h\) easily by inserting \(h_0\) and \(h_1\) below

\begin{align*}
V_T^h&=B_Th_0(T)+S_Th_1(T)\\
&=B_T\left(-e^{r(T-2T)+\sigma^2(T-T)}S_T^2\right)+S_T\left(2e^{(r+\sigma^2)(T-T)}S_T\right)\\
&=-S_T^2+2S_T^2=S_T^2.
\end{align*}

and so \(h\) hedge the payout \(\Phi(S_T)=S_T^2\). \(\square\)

\hypertarget{problem-3-1}{%
\subsubsection{Problem 3}\label{problem-3-1}}

Consider a two-dimensional model. The market model consist of three
assets: A bank account \(B_t\) and two stocks \(S_1\) and \(S_2\). The
\(P\)-dynamics of \(B_t\) is

\[
dB_t=rB_t\ dt,\ B_0=1,
\]

where \(r\in\mathbb{R}\) is a constant interest rate. The \(P\)-dynamics
of \(S_1\) and \(S_2\) are given by

\begin{align*}
dS_1(t)&=\alpha_1S_1(t)\ dt+\sigma_1S_1(t)\ d\overline{W}_1(t),&S_1(0)=s_1>0,\\
dS_2(t)&=\alpha_2S_2(t)\ dt+\sigma_2S_2(t)\ d\overline{W}_2(t),&S_2(0)=s_2>0,
\end{align*}

where \(\alpha_1,\alpha_2\in\mathbb{R}\) are constants. Moreover,
\(\sigma_1>0\) is a constant and \(\sigma_2(t)=\sigma_0e^{-\gamma t}\)
where \(\sigma_0>0\) and \(\gamma>0\) are constants and
\(\overline{W}_1(t)\) and \(\overline{W}_2(t)\) are two independent
\(P\)-Brownian motions. The filtration is the one generated by the two
Brownian motions, that is,
\(\mathcal{F}_t=\sigma(\overline{W}_1(s),\overline{W}_2(s)\ \vert\ 0\le s\le t)\).
Let \(T>0\) be a given and fixed (expiry) date.

\begin{enumerate}
\def\labelenumi{\alph{enumi}.}
\item
  \begin{enumerate}
  \def\labelenumii{\roman{enumii}.}
  \tightlist
  \item
    Is the model arbitrage free?
  \item
    Is the model complete?
  \end{enumerate}
\end{enumerate}

Consider the derivative that at time \(T\) pays \(X=S_1(T)S_2(T)\) and
let \(F(t,s_1,s_2)\) be the pricing function of the derivative.

\begin{enumerate}
\def\labelenumi{\alph{enumi}.}
\setcounter{enumi}{1}
\item
  \begin{enumerate}
  \def\labelenumii{\roman{enumii}.}
  \tightlist
  \item
    Determine the arbitrage free price of derivative \(X\) at time
    \(t=0\).
  \item
    Determine the equation satisfied by the pricing function
    \(F(t,s_1,s_2)\).
  \end{enumerate}
\end{enumerate}

Consider a new derivative that at time \(T\) pays \(Y=\log(S_2(T))\).

\begin{enumerate}
\def\labelenumi{\alph{enumi}.}
\setcounter{enumi}{2}
\tightlist
\item
  Determine the arbitrage free price of derivative \(Y\) at time
  \(t<T\).
\end{enumerate}

\textbf{Solution (a).}

\emph{(i)}: We know that the model is arbitrage free if and only if
there exist a martingale measure \(Q\). This is equivalent with finding
a likelihood process \(L\) with Radon-Nikodym derivative \(\varphi\)
given by the solution to the equation

\[
\sigma_t\varphi_t=r_t-\mu_t.
\]

We see that

\[
\sigma_t=
\begin{bmatrix}
\sigma_1 & 0\\
0 & \sigma_0e^{-\gamma t}
\end{bmatrix}\ \Rightarrow
\sigma_t^{-1}=
\begin{bmatrix}
1/\sigma_1 & 0\\
0 & e^{\gamma t}/\sigma_0
\end{bmatrix}.
\]

Hence we trivially have \textbf{a solution} given by

\[
\varphi_t=\begin{bmatrix}
1/\sigma_1 & 0\\
0 & e^{\gamma t}/\sigma_0
\end{bmatrix}\begin{bmatrix}
r-\alpha_1\\
r-\alpha_2
\end{bmatrix}=\begin{bmatrix}
\frac{r-\alpha_1}{\sigma_1}\\
\frac{r-\alpha_2}{\sigma_0}e^{\gamma t}
\end{bmatrix}.
\]

Proposition 14.1 gives now that if \(L\), given by

\[
dL_t=\varphi_t^\top L_t\ dW_t,\ L_0=1,
\]

is a martingale then the market is arbitrage free. This is true if the
Novikov condition is satisfied. We have

\[
E^P\left[e^{\frac{1}{2}\int_0^T\Vert\varphi_t\Vert^2\ dt}\right]=e^{\frac{1}{2}\int_0^T\Vert\varphi_t\Vert^2\ dt}= e^{\frac{1}{2}\int_0^T(\frac{r-\alpha_1}{\sigma_1})^2+(\frac{r-\alpha_2}{\sigma_0}e^{\gamma t})^2\ dt}<\infty
\]

since of cause

\[
\int_0^T(\frac{r-\alpha_1}{\sigma_1})^2+(\frac{r-\alpha_2}{\sigma_0}e^{\gamma t})^2\ dt=T(\frac{r-\alpha_1}{\sigma_1})^2+(\frac{r-\alpha_2}{\sigma_0})^2\int_0^Te^{2\gamma t}\ dt<\infty
\]

for all \(T\ge 0\). Then the Novikov condition is satisfied and \(L\) is
martingale with \(E[L_T]=1\). \(\square\)

\emph{(ii)}: The model is complete if the martingale measure is unique.
This is equivalent with \(Ker[\sigma_t]=\{0\}\) and since \(\sigma_t\)
is invertible (diagonal) we have that the model is complete. \(\square\)

\textbf{Solution (b).}

\emph{(i)}: We may determine the price of the derivative using the risk
neutral valueation formula

\[
\Pi_t[X]=E^Q\left[\left.e^{-\int_t^Tr(u)\ du}X\ \right\vert\ \mathcal{F}_t\right]
\]

Hence we have for \(t=0\) and \(S_1(0)=s_1\) and \(S_2(0)=s_2\) that

\[
\Pi_0[X]=E^Q\left[\left.e^{-\int_0^Tr(u)\ du}X\ \right\vert\ \mathcal{F}_0\right]=e^{-rT}E^Q\left[\left. S_1(T)S_2(T)\ \right\vert\ \mathcal{F}_0\right],
\]

Since we have that \(S_1\) and \(S_2\) have dynamics wrt. two
independent Brownian motions we know that the price processes are
independent. If we multiply by \(B(T)/B(T)\) we obtain two martingale
processes under the measure \(Q\):

\begin{align*}
\Pi_0[X]&=e^{-rT}B(T)^2E^Q\left[\left. \frac{S_1(T)}{B(T)}\ \right\vert\ \mathcal{F}_0\right]E^Q\left[\left. \frac{S_2(T)}{B(T)}\ \right\vert\ \mathcal{F}_0\right]\\
&=e^{-rT}e^{2rT}s_1(0)s_2(0)=e^{rT}s_1(0)s_2(0),
\end{align*}

and so the arbitrage free price is given above. \(\square\)

\emph{(ii)}: We have from Bjork (14.31) that \(\Pi\) satisfies the PDE
below

\[
\begin{cases}
F_t+\sum_{i=1}^2rs_iF_{s_i}+\frac{1}{2}\text{tr}[\sigma_t^\top D(S)F_{ss}D(S)\sigma_t]-rF=0\\
F(T,s_1,s_2)=\Phi(s_1,s_2)
\end{cases}
\]

The PDE is in detail

\begin{align*}
&0+rS_1(t)S_2(t)+rS_2S_1+\frac{1}{2}\text{tr}
\begin{bmatrix}
S_1\sigma_1 & 0\\
0 & S_2\sigma_0 e^{-\gamma t}
\end{bmatrix}
\begin{bmatrix}
0 & 1\\
1 & 0 
\end{bmatrix}
\begin{bmatrix}
S_1\sigma_1 & 0\\
0 & S_2\sigma_0 e^{-\gamma t}
\end{bmatrix} -r\Pi_t\\
&=2rS_1(t)S_2(t)+\frac{1}{2}\text{tr}
\begin{bmatrix}
S_1\sigma_1 & 0\\
0 & S_2\sigma_0 e^{-\gamma t}
\end{bmatrix}
\begin{bmatrix}
0 & S_2\sigma_0 e^{-\gamma t}\\
S_1\sigma_1 & 0
\end{bmatrix}\\
&=2rS_1(t)S_2(t)+\frac{1}{2}\text{tr}
\begin{bmatrix}
0 & S_1S_2\sigma_0\sigma_1e^{-\gamma t}\\
S_1S_2\sigma_0\sigma_1e^{-\gamma t} & 0
\end{bmatrix}-r\Pi_t\\
&=2rS_1(t)S_2(t)-r\Pi_t=0.
\end{align*}

or

\[
F(t,s_1,s_2)=2s_1s_2,\ F(T,s_1,s_2)=s_1s_2
\]

this ends the question. \(\square\)

\textbf{Solution (c).}

We have the derivative \(Y=\log(S_2(T))\). By the risk neutral valuation
formula we have that the arbitrage free price is given by

\[
\Pi_t[Y]=E^Q\left[\left.e^{-\int_t^Tr(u)\ du}Y\ \right\vert\ \mathcal{F}_t\right]=e^{-r(T-t)}E^Q\left[\left.\log(S_2(T))\ \right\vert\ \mathcal{F}_t\right].
\]

Under the measure \(Q\) the dynamics of \(S_2\) is that of a GBM hence

\[
d\log(S_2(t))=\left(r-\frac{1}{2}\sigma_0^2e^{-2\gamma t}\right)\ dt+\sigma_0^2e^{-2\gamma t}\ dW^Q_t,
\]

and so with the knowledge that \(S_2(t)=s_2\) we have

\begin{align*}
\Pi_t[Y]&=e^{-r(T-t)}E^Q\left[\left.\log(s_2)+\int_t^T\left(r-\frac{1}{2}\sigma_0^2e^{-2\gamma s}\right)\ ds + \int_t^T \sigma_0^2e^{-2\gamma t}\ dW^Q_t\ \right\vert\ \mathcal{F}_t\right]\\
&=e^{-r(T-t)}\left[\log(s_2)+\int_t^T\left(r-\frac{1}{2}\sigma_0^2e^{-2\gamma s}\right)\ ds\right]\\
&=e^{-r(T-t)}\left[\log(s_2)+r(T-t)-\frac{1}{2}\sigma_0^2\int_t^Te^{-2\gamma s}\ ds\right]\\
&=e^{-r(T-t)}\left[\log(s_2)+r(T-t)+\frac{1}{4\gamma}\sigma_0\left[e^{-2\gamma s}\right]_t^T\right]\\
&=e^{-r(T-t)}\left[\log(s_2)+r(T-t)+\frac{1}{4\gamma}\sigma_0(e^{-2\gamma T}-e^{-2\gamma t})\right].
\end{align*}

The arbitrage free price of the derivative is then given above.
\(\square\)

\hypertarget{exam-201819}{%
\subsection{Exam 2018/19}\label{exam-201819}}

\hypertarget{problem-1-1}{%
\subsubsection{Problem 1}\label{problem-1-1}}

Let \(W(t)\) denote a Brownian motion and let
\(\mathcal{F}_t=\mathcal{F}_t^W\). Let \(T>0\) be a given and fixed
time.

Consider the stochastic differential equation

\[
dX(t)=\alpha\ dt+\sqrt{X(t)}\ dW(t),
\]

and \(X(0)=x>0\) where \(\alpha\in\mathbb{R}\).

\begin{enumerate}
\def\labelenumi{\alph{enumi}.}
\item
  \begin{enumerate}
  \def\labelenumii{\roman{enumii}.}
  \tightlist
  \item
    Compute the mean value of \(X(T)\).
  \item
    Compute the variance of \(X(T)\).
  \end{enumerate}
\item
  Find the solution of the partial differential equation \begin{align*}
    &4F_t(t,x)+8x^2F_{xx}(t,x)+3xF_x(t,x)=5F(t,x)\ \text{for}\ t<T\ \text{and}\ x>0.\\
    &F(T,x)=x^3.
    \end{align*}
\end{enumerate}

Let \(\widetilde{W}(t)\) be another Brownian motion such that \(W(t)\)
and \(\widetilde{W}(t)\) are two independent Brownian motions. Let
\(Y(t)\) and \(Z(t)\) be two martingales given by the following dynamics

\begin{align*}
dY(t)&=W(t)\ dW(t)+\widetilde{W}(t)\ d\widetilde{W}(t),\\
dZ(t)&=\widetilde{W}(t)\ dW(t)-W(t)\ d\widetilde{W}(t).
\end{align*}

with \(Y(0)=Z(0)=0\).

\begin{enumerate}
\def\labelenumi{\alph{enumi}.}
\setcounter{enumi}{2}
\tightlist
\item
  Show that \(M(t)=Y(t)Z(t)\) is a martingale.
\end{enumerate}

\textbf{Solution (a).}

\emph{(i)}: We start by writing \(X\) on integral form given as

\[
X(t)=x+\int_0^t\alpha\ dt+\int_0^t\sqrt{X(s)}\ dW(s).
\]

Taking expectation yields.

\[
E[X(t)]=E\left[x+\alpha t+\int_0^t\sqrt{X(s)}\ dW(s)\right]=x+\alpha t,
\]

since

\[
E\left[\int_0^t\sqrt{X(s)}\ dW(s)\right]=E\left[\int_0^0\sqrt{X(s)}\ dW(s)\right]=0.
\]

This result follows from lemma 4.10 as the process
\(M_t=\int_0^t\sqrt{X(s)}\ dW(s)\) is a martingale. From this we have
that \(E[X(T)]=x+\alpha T\). \(\square\)

\emph{(ii)}: We have that the variance is given by

\[
Var(X(t))=E(X^2(t))-(E(X(t))^2.
\]

and so we have

\begin{align*}
Var(X(t))+E(X(t))^2&=E\left[\left(x+t\alpha+\int_0^t\sqrt{X(s)}\ dW(s)\right)^2\right]\\
&=(x+\alpha t)^2+E\left[\left(\int_0^t\sqrt{X(s)}\ dW(s)\right)^2\right]+2(x+\alpha t)E\left[\int_0^t\sqrt{X(s)}\ dW(s)\right]\\
&=(x+\alpha t)^2+E\left[\left(\int_0^t\sqrt{X(s)}\ dW(s)\right)^2\right].
\end{align*}

Now by setting \(Z(t)= \int_0^t\sqrt{X(s)}\ dW(s)\) we see that \(Z\)
has dynamics \(dZ(t)=\sqrt{X(t)}\ dW(t)\) with \(Z(0)=0\). By Ito's
formula on the variable \(f(t,Z(t))\) with \(f(t,z)=z^2\) we have

\begin{align*}
df(t,Z(t))&=0\ dt+2Z(t)\ dZ(t)+\frac{1}{2}2\ (dZ(t))^2\\
&=2Z(t)\sqrt{X(t)}\ dW(t)+X(t)\ dt.
\end{align*}

Obviously when taking expectation on \(f(t,Z(t))\) we see that the
integral part related to the Brownian motion is a martingale with mean 0
and then

\[
E[f(t,Z(t))]=E\left[\int_0^t X(s)\ ds\right].
\]

In total we have

\[
Var(X(t))=(x+\alpha t)^2+E\left[\int_0^t X(s)\ ds\right]-(x+\alpha t)^2=E\left[\int_0^t X(s)\ ds\right].
\]

Moving the expectation inside the integral then gives

\[
Var(X(t))=\int_0^t(x+\alpha s)\ ds=xt+\frac{1}{2}\alpha t^2.
\]

Inserting \(t=T\) gives the desired result. \(\square\)

\textbf{Solution (b).}

We see by dividing by 4 we have the PDE given by

\[
F_t+2x^2F_{xx}+\frac{3}{4}xF_x=\frac{5}{4}F
\]

hence by setting \(r=5/4\), \(\mu=3x/4\) and \(\sigma^2=4x^2\) we have
the boundary value problem

\[
\begin{cases}
F_t+\mu F_x+\frac{1}{2}\sigma ^2F_{xx}-rF=0,\\
F(T,x)=x^3.
\end{cases}
\]

From Feymann-Kac we know this has solution on \([0,T]\times\mathbb{R}\)
given by the stochastic representation

\[
F(t,x)=e^{-r(T-t)}E_{t,x}[X_T^3],
\]

where \(X\) satisfies the SDE

\[
dX_t=\frac{3}{4}X_t\ dt+2X_t\ dW_t.
\]

Giving that \(X(t)=x\) and \(X\) is a GBM we have

\[
X_T=x\cdot e^{\left(r-\frac{1}{2}2^2\right)(T-t)+2(W_T-W_t)}=x\cdot e^{\frac{-5}{4}(T-t)+2(W_T-W_t)}.
\]

The relevant mean value is then

\begin{align*}
F(t,x)&=e^{-\frac{5}{4}(T-t)}E\left[x^3\cdot e^{\frac{-15}{4}(T-t)+6(W_T-W_t)}\right]\\
&=e^{-\frac{5}{4}(T-t)}x^3e^{\frac{-15}{4}(T-t)}E\left[e^{6(W_T-W_t)}\right]\\
&=x^3e^{\frac{-20}{4}(T-t)}e^{\frac{1}{2}6^2(T-t)}=x^3e^{-5(T-t)+18(T-t)}\\
&=x^3e^{13(T-t)}.
\end{align*}

The solution is the given above. \(\square\)

\textbf{Solution (c).}

We show that \(M\) has dynamics solely given in terms of Brownian
motions. We have that \(M(t)=f(t,Y(t),Z(t))\) for \(f(t,y,z)=yz\) the
dynamics given by Ito's formula:

\[
dM(t)=0\ dt+Z(t)\ dY(t)+Y(t)\ dZ(t)+(dY(t))(dZ(t))
\]

since the only second derivative not zero is \(f_{yz}=f_{zy}=1\). The
product \((dY(t))(dZ(t))\) is computed first

\begin{align*}
(dY(t))(dZ(t))&=(W(t)\ dW(t)+\widetilde{W}(t)\ d\widetilde{W}(t))\cdot(\widetilde{W}(t)\ dW(t)-W(t)\ d\widetilde{W}(t))\\
&=W(t)\widetilde{W}(t)\ dt-\widetilde{W}(t)W(t)\ dt=0,
\end{align*}

where we use that \(dW(t)d\widetilde{W}(t)=dt\) is the only non-zero
term. Then we obviously have

\begin{align*}
dM(t)&=Z(t)\ dY(t)+Y(t)\ dZ(t)\\
&=Z(t)W(t)\ dW(t)+Z(t)\widetilde{W}(t)\ d\widetilde{W}(t)+Y(t)\widetilde{W}(t)\ dW(t)-Y(t)W(t)\ d\widetilde{W}(t).
\end{align*}

Giving that \(M(t)\) is a martingale. (lemma 4.11)

\hypertarget{problem-2-2}{%
\subsubsection{Problem 2}\label{problem-2-2}}

Consider a standard Black-Scholes model, that is, a model consisting of
a bank account \(B(t)\) with \(P\)-dynamics given by

\[
dB(t)=rB(t)\ dt,
\]

with \(B(0)=1\) and a stock \(S(t)\) with \(P\)-dynamics given by

\[
dS(t)=\alpha S(t)\ dt+\sigma S(t)\ d\overline{W}(t),
\]

with \(S(0)=s>0\) and where \(r,\alpha\in\mathbb{R}\) and \(\sigma >0\)
are constants and \(\overline{W}(t)\) is a \(P\)-Brownian motion. Let
\(T>0\) be a given fixed (expiry) date.

Let \(h(t)=\left(h_0(t),h_1(t)\right)\) be a portfolio where

\[
h_0(t)=\exp\left(\frac{1}{2}\sigma\overline{W}(t)+\left(\frac{\alpha - r}{2}-\frac{1}{8}\sigma^2\right)t\right)
\]

is the number of units in the bank account at time \(t\) and

\[
h_1(t)=\frac{1}{s}\exp\left(-\frac{1}{2}\sigma\overline{W}(t)+\left(\frac{r-\alpha}{2}-\frac{3}{8}\sigma^2\right)t\right)
\]

is the number of shares in the stock at time \(t\). Let \(V^h(t)\)
denote the associated value process and let
\(u(t)=\left(u_0(t),u_1(t)\right)\) denote the relative portfolio.

\begin{enumerate}
\def\labelenumi{\alph{enumi}.}
\item
  \begin{enumerate}
  \def\labelenumii{\roman{enumii}.}
  \tightlist
  \item
    Determine whether the portfolio \(h\) is self-financing or not.
  \item
    Compute \(u_1(t)\).
  \end{enumerate}
\end{enumerate}

Consider two derivatives that at time \(T\) pay \(X_1=\Phi_1(S(T))\) and
\(X_2=\Phi_2(S(T))\). For \(i=1,2\), the arbitrage free price of
derivative \(X_i\) is given by \(\pi_i(t)=F_i(t,S(t))\) where
\(F_i(t,s)\) is the pricing function of the derivative. Assume that
\(\pi_i(t)>0\). The price process \(\pi_i(t)\) has dynamics (under the
\(P\)-measure) given by

\[
d\pi_i(t)=\alpha_i(t)\pi_i(t)\ dt+\sigma_i(t)\pi_i(t)\ d\overline{W}(t).
\]

\begin{enumerate}
\def\labelenumi{\alph{enumi}.}
\setcounter{enumi}{1}
\item
  \begin{enumerate}
  \def\labelenumii{\roman{enumii}.}
  \tightlist
  \item
    Determine \(\alpha_i(t)\) and \(\sigma_i(t)\) for \(i=1,2\).
  \item
    Show that \[
    \frac{r-\alpha_1(t)}{\sigma_1(t)}=\frac{r-\alpha_2(t)}{\sigma_2(t)}.
    \]
  \end{enumerate}
\end{enumerate}

Let \(C(t,s;K,T)\) denote the Black-Scholes price at time \(t\) of an
European call option with strike \(K\) and expiry date \(T\) when the
current price of the underlying is \(s\). Similary, let \(P(t,s;K,T)\)
denote the Black-Scholes price at time \(t\) of an European put option
with strike \(K\) and expiry date \(T\) when the current price of the
underlying is \(s\). Consider a new derivative that at time \(T\) pays

\[
Y=\max\left\{C(T,S(T);K,T_1),P(T,S(T);K,T_1\right\}
\]

where \(T<T_1\) is a fixed date.

\begin{enumerate}
\def\labelenumi{\alph{enumi}.}
\setcounter{enumi}{2}
\tightlist
\item
  Determine the arbitrage free price of derivative \(Y\) at time
  \(t<T\). (Hint: recall \(\max(x,y)=(x+y)^+y\))
\end{enumerate}

Assume that the call option and the put option do not have the same
strike prices, that is, a derivative that at time \(T\) pays

\[
\widetilde{Y}=\max\left\{C(T,S(T);K_1,T_1),P(T,S(T);K_2,T_1\right\}
\]

where the strike prices \(K_1\ne K_2\). Let \(F(t,s)\) be the pricing
function of the derivative.

\begin{enumerate}
\def\labelenumi{\alph{enumi}.}
\setcounter{enumi}{3}
\tightlist
\item
  Determine the equation satisfied by the pricing function \(F(t,s)\).
\end{enumerate}

\textbf{Solution (a).}

We have that \(h\) is self-financing if and only if the equation

\[
dV^h(t)=h_0(t)\ dB(t)+h_1(t)\ dS(t)
\]

is satisfied. And so, we start by determining the dynamics of the number
of assets denoted by \(h_0\) and \(h_1\). From Ito's formula we can
conclude that

\begin{align*}
dh_0(t)&=\left(\frac{\alpha - r}{2}-\frac{1}{8}\sigma^2\right)h_0(t)\ dt+\frac{1}{2}\sigma h_0(t)\ dW(t)+\frac{1}{2} \frac{1}{2}\sigma \frac{1}{2}\sigma h_0 \ (d W(t))^2\\
&=\left(\frac{\alpha - r}{2}-\frac{1}{8}\sigma^2\right)h_0(t)\ dt+\frac{1}{2}\sigma h_0(t)\ dW(t)+ \frac{1}{2^3}\sigma^2 h_0 \ dt\\
&=\left(\frac{\alpha - r}{2}-\frac{1}{8}\sigma^2+ \frac{1}{8}\sigma^2 \right)h_0(t)\ dt+\frac{1}{2}\sigma h_0(t)\ dW(t)\\
&=\frac{\alpha - r}{2}h_0(t)\ dt+\frac{1}{2}\sigma h_0(t)\ dW(t).
\end{align*}

For the number of stocks we have

\begin{align*}
dh_1(t)&=\left(\frac{r-\alpha}{2}+\frac{3}{8}\sigma^2\right)h_1(t)\ dt-\frac{1}{2}\sigma h_1(t)\ dW(t)+ \frac{1}{2}\frac{1}{2}\sigma\frac{1}{2}\sigma h_1(t)\ (dW(t))^2\\
&=\left(\frac{r-\alpha}{2}+\frac{1}{2}\sigma^2\right)h_1(t)\ dt-\frac{1}{2}\sigma h_1(t)\ dW(t).
\end{align*}

We may derive the dynamics of the portfolio as

\begin{align*}
dV^h(t)&=d(h_0(t)B(t)+h_1(t)S(t))\\
&=B(t)\ dh_0(t)+h_0(t)\ dB(t)+(dh_0(t))(dB(t))\\
&+S(t)\ dh_1(t)+h_1(t)\ dS(t)+(dh_1(t))(dS(t))
\end{align*}

and so we want that

\[
(*)=B(t)\ dh_0(t)+(dh_0(t))(dB(t))+S(t)\ dh_1(t)+(dh_1(t))(dS(t))=0.
\]

Inserting the dynamics given and portfolio dynamics above we have

\begin{align*}
(*)&=B(t)\frac{\alpha - r}{2}h_0(t)\ dt+B(t)\frac{1}{2}\sigma h_0(t)\ dW(t)+0\\
&+S(t)\left(\frac{r-\alpha}{2}+\frac{1}{2}\sigma^2\right)h_1(t)\ dt-S(t)\frac{1}{2}\sigma h_1(t)\ dW(t)\\
&-\frac{1}{2}\sigma h_1(t)\sigma S(t) \ dt\\
&=\left(B(t)\frac{\alpha - r}{2}h_0(t)-\frac{\alpha-r}{2}S(t)h_1(t)\right)\ dt\\
&+\left(\frac{1}{2}B(t)\sigma h_0(t)-\frac{1}{2}S(t)\sigma h_1(t)\right)\ dW(t)
\end{align*}

We see that this is zero if \(h_0(t)B(t)=h_1(t)S(t)\). First we have

\begin{align*}
h_0(t)B(t)&=\exp\left(\frac{1}{2}\sigma\overline{W}(t)+\left(\frac{\alpha - r}{2}-\frac{1}{8}\sigma^2\right)t\right)B(t)\\
&=\exp\left(\frac{1}{2}\left(\alpha - r-\frac{1}{4}\sigma^2\right)t+\frac{1}{2}\sigma\overline{W}(t)\right)B(t)\\
&=\exp\left(\frac{1}{2}\left(\alpha-\frac{1}{2}\sigma^2\right)t+\frac{1}{2}\sigma\overline{W}(t)\right)\exp\left(\frac{1}{2}\left( - r+\frac{1}{4}\sigma^2\right)t\right)B(t)\\
&=(S(t))^{1/2}\exp\left(\left(-\frac{r}{2}+\frac{1}{8}\sigma^2\right)t\right)B(t),
\end{align*}

and

\begin{align*}
h_1(t)S(t)&=\frac{1}{s}\exp\left(-\frac{1}{2}\sigma\overline{W}(t)+\left(\frac{r-\alpha}{2}-\frac{3}{8}\sigma^2\right)t\right)s\cdot\exp\left(\left(\alpha-\frac{1}{2}\sigma^2\right)t+\sigma\overline{W}(t)\right)\\
&=\exp\left(\frac{1}{2}\sigma\overline{W}(t)+\left(\frac{r+\alpha}{2}-\frac{7}{8}\sigma^2\right)t\right)\\
&=\exp\left(\frac{1}{2}\sigma\overline{W}(t)+\frac{1}{2}\left(\alpha-\frac{2}{4}\sigma^2\right)t\right)\exp\left(\frac{1}{2}\left(r-\frac{5}{4}\sigma^2\right)t\right)\\
&=(S(t))^{1/2}\exp\left(\left(-\frac{5}{8}\sigma^2\right)t\right)B(t)
\end{align*}

Which does not hold. \textbf{THIS EXERCISE SHOULD BE ABLE TO BE
SOLVED..} \(\square\)

\emph{(ii)}: We have that

\[
u_1(t)=\frac{h_1(t)S(t)}{V^h(t)}.
\]

Using that \(S\) is a GBM and \(B(t)=e^{rt}\) we have

\begin{align*}
u_1(t)&=\frac{h_1(t)S(t)}{h_1(t)S(t)+h_0(t)B(t)}\\
&=\frac{e^{-\frac{1}{2}\sigma\overline{W}(t)+\left(\frac{r-\alpha}{2}-\frac{3}{8}\sigma^2\right)t}e^{\left(\alpha-\frac{1}{2}\sigma^2\right)t+\sigma \overline{W}_t}}{e^{-\frac{1}{2}\sigma\overline{W}(t)+\left(\frac{r-\alpha}{2}-\frac{3}{8}\sigma^2\right)t}e^{\left(\alpha-\frac{1}{2}\sigma^2\right)t+\sigma \overline{W}_t}+e^{\frac{1}{2}\sigma\overline{W}(t)+\left(\frac{\alpha - r}{2}-\frac{1}{8}\sigma^2\right)t}e^{rt}}\\
&=\frac{se^{\left(\frac{r-\alpha}{2}-\frac{3}{8}\sigma^2+\alpha-\frac{1}{2}\sigma^2\right)t+\frac{1}{2}\sigma \overline{W}_t}}{e^{\left(\frac{r-\alpha}{2}-\frac{3}{8}\sigma^2+\alpha-\frac{1}{2}\sigma^2\right)t+\frac{1}{2}\sigma \overline{W}_t}+e^{\frac{1}{2}\sigma\overline{W}(t)+\left(\frac{\alpha - r}{2}-\frac{1}{8}\sigma^2+r\right)t}}\\
&=\frac{e^{\left(\frac{r+\alpha}{2}-\frac{7}{8}\sigma^2\right)t+\frac{1}{2}\sigma \overline{W}_t}}{e^{\left(\frac{r+\alpha}{2}-\frac{7}{8}\sigma^2\right)t+\frac{1}{2}\sigma \overline{W}_t}+e^{\frac{1}{2}\sigma\overline{W}(t)+\left(\frac{\alpha + r}{2}-\frac{1}{8}\sigma^2\right)t}}\\
&=\frac{e^{-\frac{7}{8}\sigma^2t}}{e^{-\frac{7}{8}\sigma^2t}+e^{-\frac{1}{8}\sigma^2t}}=\frac{e^{-\frac{6}{8}\sigma^2t}}{e^{-\frac{6}{8}\sigma^2t}+1}.
\end{align*}

\textbf{OBVIOUSLY} had the previous exercise been done correct we would
have \(h_1(t)S(t)=h_0(t)B(t)\) i.e.~\(u_1(t)=\frac{1}{2}\). \(\square\)

\textbf{Solution (b).}

\emph{(i)}: We know that \(\pi_i(t)=F_i(t,S(t))\) and so from Ito's
formula we have the dynamics (we suppress the argument \((t,S(t))\) in
the derivatives):

\begin{align*}
d\pi_i(t)&=\frac{\partial F_i}{\partial t}\ dt+\frac{\partial F_i}{\partial s}\ dS(t)+\frac{1}{2}\frac{\partial^2 F_i}{\partial s^2}\ (dS(t))^2\\
&=\frac{\partial F_i}{\partial t}\ dt+\frac{\partial F_i}{\partial s}\ (\alpha S(t)\ dt+\sigma S(t)\ d\overline{W}(t))+\frac{1}{2}\frac{\partial^2 F_i}{\partial s^2}\ \sigma^2 S(t)^2\ dt\\
&=\left(\frac{\partial F_i}{\partial t}+\frac{\partial F_i}{\partial s}\alpha S(t)+\frac{1}{2}\frac{\partial^2 F_i}{\partial s^2}\ \sigma^2 S(t)^2\right)\ dt+\frac{\partial F_i}{\partial s}\sigma S(t)\ d\overline{W}(t)\\
&=\underbrace{\frac{\frac{\partial F_i}{\partial t}+\frac{\partial F_i}{\partial s}\alpha S(t)+\frac{1}{2}\frac{\partial^2 F_i}{\partial s^2}\ \sigma^2 S(t)^2}{\pi_i(t)}}_{=\alpha_i(t)}\pi_i(t)\ dt+\underbrace{\frac{\frac{\partial F_i}{\partial s}\sigma S(t)}{\alpha_i(t)}}_{=\sigma_i(t)}\pi_i(t)\ d\overline{W}(t)
\end{align*}

as desired. \(\square\)

\emph{(ii)}: We have

\begin{align*}
\frac{r-\alpha_i(t)}{\sigma_i(t)}&=\frac{r-\frac{\frac{\partial F_i}{\partial t}+\frac{\partial F_i}{\partial s}\alpha S(t)+\frac{1}{2}\frac{\partial^2 F_i}{\partial s^2}\ \sigma^2 S(t)^2}{\pi_i(t)}}{\frac{\frac{\partial F_i}{\partial s}\sigma S(t)}{\alpha_i(t)}}\\
&=\frac{r\pi_i(t)-\frac{\partial F_i}{\partial t}-\frac{\partial F_i}{\partial s}\alpha S(t)-\frac{1}{2}\frac{\partial^2 F_i}{\partial s^2}\ \sigma^2 S(t)^2}{\frac{\partial F_i}{\partial s}\sigma S(t)}\\
&=\frac{r\pi_i(t)-\frac{\partial F_i}{\partial t}-\frac{\partial F_i}{\partial s}r S(t)-\frac{1}{2}\frac{\partial^2 F_i}{\partial s^2}\ \sigma^2 S(t)^2+\frac{\partial F_i}{\partial s}r S(t)-\frac{\partial F_i}{\partial s}\alpha S(t)}{\frac{\partial F_i}{\partial s}\sigma S(t)}\\
&=\frac{\frac{\partial F_i}{\partial s}r S(t)-\frac{\partial F_i}{\partial s}\alpha S(t)}{\frac{\partial F_i}{\partial s}\sigma S(t)}=\frac{r -\alpha }{\sigma },
\end{align*}

where we used the Black-Scholes equation i.e.

\[
r\pi_i(t)-\frac{\partial F_i}{\partial t}-\frac{\partial F_i}{\partial s}r S(t)-\frac{1}{2}\frac{\partial^2 F_i}{\partial s^2}\ \sigma^2 S(t)^2=0
\]

for any derivative's arbitrage free pricing process. Since \(i\) is not
included in the fraction above we have the desired result. \(\square\)

\textbf{Solution (c).}

We follow the hint and see that the payout is

\begin{align*}
Y&=\max\left\{C(T,S(T);K,T_1),P(T,S(T);K,T_1)\right\}\\
&=\Big(C(T,S(T);K,T_1)-P(T,S(T);K,T_1)\Big)^++P(T,S(T);K,T_1)\\
&=\Big(C(T,S(T);K,T_1)-Ke^{-r(T_1-T)}-C(T,S(T);K,T_1)+S(T)\Big)^++P(T,S(T);K,T_1)\\
&=\Big(S(T)-Ke^{-r(T_1-T)}\Big)^++P(T,S(T);K,T_1)\\
&=C(T,S(T);Ke^{-r(T_1-T)},T)+P(T,S(T);K,T_1)
\end{align*}

Hence we can hedge this payout with a call option with strike
\(Ke^{-r(T_1-T)}\) at expiry \(T\) and a put with strike \(K\) at expiry
\(T_1\), that is

\[
\Pi_t[Y]=C(t,S(t);Ke^{-r(T_1-T)},T)+P(t,S(t);K,T_1)
\]

as desired. \(\square\)

\textbf{Solution (d).}

We have that the arbitrage free pricing function \(F(t,s)\) has to
satisfie the Black-Scholes equation 7.10 i.e.

\begin{align*}
F_t+rsF_s+\frac{1}{2}\sigma ^2 s^2F_{ss}-rF=0&\\
F(T,s)=\max\left\{C(T,s;K_1,T_1),P(T,s;K_2,T_1\right\}&.
\end{align*}

which may be written differently in terms of call options, stock price
\(s\) and zero-coupon bonds. \(\square\)

\hypertarget{problem-3-2}{%
\subsubsection{Problem 3}\label{problem-3-2}}

Consider a two-dimensional Black-Scholes model. The market model consist
of three assets: A bank account \(B(t)\) and two stocks \(S_1(t)\) and
\(S_2(t)\). The \(P\)-dynamics of \(B(t)\) is

\[
dB(t)=rB(t)\ dt
\]

with \(B(0)=1\) where \(r\in\mathbb{R}\) is a constant interest rate.
The \(P\)-dynamics of \(S_1(t)\) and \(S_2(t)\) are given by

\begin{align*}
dS_1(t)&=\alpha_1S_1(t)\ dt+\sigma S_1(t)\ d\overline{W}_1(t),\\
dS_2(t)&=\alpha_2S_2(t)\ dt+\sigma S_2(t)\ \big(d\overline{W}_1(t)+d\overline{W}_2(t)\big),
\end{align*}

with \(S_1(0)=s_1>0\) and \(S_2(0)=s_2>0\) where
\(\alpha_1,\alpha_2\in\mathbb{R}\) and \(\sigma>0\) are constants and
\(\overline{W}_1(t)\) and \(\overline{W}_2(t)\) are two independent
\(P\)-Brownian motions. The filtration is the one generated by the two
Brownian motions. Let \(T>0\) be a given and fixed (expiry) date.

\begin{enumerate}
\def\labelenumi{\alph{enumi}.}
\item
  \begin{enumerate}
  \def\labelenumii{\roman{enumii}.}
  \tightlist
  \item
    Is the model arbitrage free?
  \item
    Is the model complete?
  \end{enumerate}
\item
  Compute the covariance of \(S_1(T)\) and \(S_2(T)\). (Hint: recall
  \(cov(X,Y)=E[XY]-E[X]E[Y]\)).
\end{enumerate}

Consider the derivative that at time \(T\) pays \(X=S_1(T_0)+S_2(T)\)
where \(0<T_0<T\) is a fixed date.

\begin{enumerate}
\def\labelenumi{\alph{enumi}.}
\setcounter{enumi}{2}
\tightlist
\item
  Find a hedge portfolio for derivative \(X\).
\end{enumerate}

\textbf{Solution (a).}

\emph{(i)}: The model is arbitrage free if and only if a martingale
measure exists. That is if the equation

\[
\sigma_t\varphi_t=r-\alpha
\]

has at least one solution. We have the following market on matrix form

\[
dS_t=D(S_t)\alpha_t\ dt+D(S_t)\sigma _t\ dW_t
\]

or written out in total

\[
\begin{bmatrix}
dS_1(t)\\
dS_2(t)
\end{bmatrix}
=
\begin{bmatrix}
S_1(t) & 0\\
0 & S_2(t)
\end{bmatrix}
\begin{bmatrix}
\alpha_1\\
\alpha_2
\end{bmatrix}
\begin{bmatrix}
dt\\
dt
\end{bmatrix}
+
\begin{bmatrix}
S_1(t) & 0\\
0 & S_2(t)
\end{bmatrix}
\begin{bmatrix}
\sigma & 0\\
\sigma & \sigma
\end{bmatrix}
\begin{bmatrix}
d\overline{W}_1(t)\\
d\overline{W}_2(t)
\end{bmatrix}.
\]

Hence we want to solve

\[
\begin{bmatrix}
\sigma & 0\\
\sigma & \sigma
\end{bmatrix}\begin{bmatrix}
\varphi_1(t)\\
\varphi_2(t)
\end{bmatrix}=
\begin{bmatrix}
r-\alpha_1\\
r-\alpha_2
\end{bmatrix}.
\]

This is easy if \(\sigma\) is invertible. We see that we in fact have
that the inverse of \(\sigma\) is

\[
\sigma_t^{-1}=\begin{bmatrix}
1/\sigma & 0\\
-1/\sigma & 1/\sigma
\end{bmatrix}
\]

as we have

\[
\begin{bmatrix}
1/\sigma & 0\\
-1/\sigma & 1/\sigma
\end{bmatrix}
\begin{bmatrix}
\sigma & 0\\
\sigma & \sigma
\end{bmatrix}=
\begin{bmatrix}
1 & 0\\
0 & 1
\end{bmatrix}=I.
\]

Then we clearly have the solution

\[
\varphi_t=
\begin{bmatrix}
1/\sigma & 0\\
-1/\sigma & 1/\sigma
\end{bmatrix}\begin{bmatrix}
r-\alpha_1\\
r-\alpha_2
\end{bmatrix}=
\begin{bmatrix}
\frac{r-\alpha_1}{\sigma}\\
\frac{-r+\alpha_1+r-\alpha_2}{\sigma}
\end{bmatrix}=\begin{bmatrix}
\frac{r-\alpha_1}{\sigma}\\
\frac{\alpha_1-\alpha_2}{\sigma}
\end{bmatrix}.
\]

By defining the likelihood process \(L_t\) as

\[
dL_t=\varphi_t^\top L_t\ d\overline{W}_t,\ L_0=1,
\]

we know from the Novikov condition that if the integral
\(E^P[e^{1/2\int_0^T\Vert \varphi_t\Vert^2\ dt}]\) is finite then \(L\)
is a martingale with \(E^P[L_T]=1\). We see that

\[
E^P\left[e^{\frac{1}{2}\int_0^T\Vert \varphi_t\Vert^2\ dt}\right]=E^P\left[e^{\frac{1}{2}\int_0^T \left(\frac{r-\alpha_1}{\sigma}\right)^2+\left(\frac{\alpha_1-\alpha_2}{\sigma}\right)^2\ dt}\right]=E^P\left[e^{\frac{1}{2}T \left(\frac{r-\alpha_1}{\sigma}\right)^2+\frac{1}{2}T\left(\frac{\alpha_1-\alpha_2}{\sigma}\right)^2}\right]<\infty.
\]

Hence we have found a martingale measure defined by the likelihood
process \(L\) above. We conclude that the market is arbitrage free.
\(\square\)

\emph{(ii)}: The model is complete if the martingale measure is unique.
This is equivalent with \(Ker[\sigma_t]=\{0\}\) and since \(\sigma_t\)
is invertible (diagonal) we have from theorem 14.7 that the model is
complete. \(\square\)

\textbf{Solution (b).}

We have by definition:

\[
cov(S_1(T),S_2(T))=E\left[S_1(T)S_2(T)\right]-E[S_1(T)]E[S_2(T)].
\]

Thus we set \(Z(t)=S_1(t)S_2(t)\) and evaluate the mean value of \(Z\).
By Ito's formula on \(f(s_1,s_2)=s_1s_2\) we have

\begin{align*}
dZ(t)&=df(S_1(t),S_2(t))\\
&=S_2(t)\ dS_1(t)+S_1(t)\ dS_2(t)+\frac{1}{2}(dS_1(t))(dS_2(t))\\
&=S_2(t)\alpha_1(t) S_1(t)\ dt+S_2(t)\sigma S_1(t)\ d\overline{W}_1(t)\\
&+S_1(t)\alpha_2(t) S_2(t)\ dt+S_1(t)\sigma S_2(t)\ (d\overline{W}_1(t)+d\overline{W}_2(t))\\
&+\frac{1}{2}\sigma^2S_1(t)S_2(t)\ d\overline{W}_1(t)(d\overline{W}_1(t)+d\overline{W}_2(t))\\
&=(\alpha_1(t)+\alpha_2(t))S_1(t) S_2(t)\ dt+2\sigma S_1(t) S_2(t)\ d\overline{W}_1(t) + \sigma S_1(t) S_2(t)\ d\overline{W}_2(t)\\
&+\frac{1}{2}\sigma^2S_1(t)S_2(t)\ dt\\
&=(\alpha_1(t)+\alpha_2(t)+\frac{1}{2}\sigma^2)Z(t)\ dt+2\sigma Z(t)\ d\overline{W}_1(t) + \sigma Z(t)\ d\overline{W}_2(t).
\end{align*}

Thus we have that the terms invovling the Brownian motions will vanish
when takings expectation hence

\begin{align*}
E[Z(t)]&=Z(0)+E\left[\int_0^t(\alpha_1(t)+\alpha_2(t)+\frac{1}{2}\sigma^2)Z(s)\ ds\right]\\
&=Z(0)+(\alpha_1(t)+\alpha_2(t)+\frac{1}{2}\sigma^2)\int_0^tE\left[Z(s)\right]\ ds.
\end{align*}

Then we have the dynamics of \(E[Z(t)]\) is given as

\[
dE[Z(t)]=(\alpha_1(t)+\alpha_2(t)+\frac{1}{2}\sigma^2)E\left[Z(t)\right]\ dt.
\]

We may then solve this using \(E[Z(0)]=Z(0)=s_1s_2\):

\[
E[Z(T)]=Z(0)e^{(\alpha_1(t)+\alpha_2(t)+\frac{1}{2}\sigma^2)T}=s_1s_2e^{(\alpha_1(t)+\alpha_2(t)+\frac{1}{2}\sigma^2)T}.
\]

Inserting in the formula for covariance we arrive at

\begin{align*}
cov(S_1(T),S_2(T))&=E\left[S_1(T)S_2(T)\right]-E[S_1(T)]E[S_2(T)]\\
&=s_1s_2e^{(\alpha_1(t)+\alpha_2(t)+\frac{1}{2}\sigma^2)T}-E[S_1(T)]E[S_2(T)]\\
&=s_1s_2e^{(\alpha_1+\alpha_2+\frac{1}{2}\sigma^2)T}-s_1e^{\alpha_1T}s_2^{\alpha_2T}\\
&=s_1s_2e^{(\alpha_1+\alpha_2)T}\left(e^{\frac{1}{2}\sigma^2T}-1\right).
\end{align*}

as desired. \(\square\)

\textbf{Solution (c).}

We may look at this problem on two subintervals: \([0,T_0]\) and
\((T_0,T]\). On the latter we know that the portfolio should consist of
\(S_1(T_0)\) zero coupon bonds with expiry \(T\) and one position in the
second stock. Hence on the interval \((T_0,T]\) the hedging portfolio is

\[
h(t)=\Big(h_0(t),h_1(t),h_2(t)\Big)=\Big(e^{-r(T-T_0)}S(T_0),0,1\Big),\ t> T_0.
\]

Hence we on the interval \([0,T_0]\) we want to replicate the derivative
\(\widetilde{X}=e^{-r(T-T_0)}S(T_0)\). This is obviously easy since we
should hold \(e^{-r(T-T_0)}\) of the first stock. Then we have

\[
h(t)=\begin{cases}
\Big(0,e^{-r(T-T_0)},1\Big) &\text{for}\ t\le T_0,\\
\Big(e^{-r(T-T_0)}S(T_0),0,1\Big) &\text{for}\ t>T_0.
\end{cases}
\]

This then give a self-financing portfolio with value process

\[
V^h(t)=\begin{cases}
S_1(t)e^{-r(T-T_0)}+S_2(t) &\text{for}\ t\le T_0,\\
e^{-r(T-t)}S(T_0)+S_2(t) &\text{for}\ t>T_0.
\end{cases}
\]

as desired. \(\square\)

\hypertarget{exam-201920}{%
\subsection{Exam 2019/20}\label{exam-201920}}

\hypertarget{problem-1-2}{%
\subsubsection{Problem 1}\label{problem-1-2}}

Let \(W(t)\) denote a Brownian motion and let
\(\mathcal{F}_t=\mathcal{F}_t^W\). Let \(T>0\) be a given and fixed
time.

Consider the two dimensional stochastic differential equation

\begin{align*}
dX(t)&=\frac{1}{2}X(t)\ dt+Y(t)\ dW(t),\\
dY(t)&=\frac{1}{2}Y(t)\ dt+X(t)\ dW(t),
\end{align*}

with \(X(0)=0\) and \(Y(1)=1\).

\begin{enumerate}
\def\labelenumi{\alph{enumi}.}
\item
  Show that \((X(t),Y(t))=(\text{sinh}(W(t)),\text{cosh}(W(t)))\) solves
  the two-dimensional stochastic differential equation. (Hint: Recall
  that \(\text{sinh}(x)=\frac{1}{2}(e^x-e^{-x})\) and
  \(\text{cosh}(x)=\frac{1}{2}(e^x+e^{-x})\)).
\item
  \begin{enumerate}
  \def\labelenumii{\roman{enumii}.}
  \tightlist
  \item
    Show that \(M(t)=e^{-t/2}\text{cosh}(W(t))\) is a martingale.
  \item
    Find a constant \(z\) and a process \(h(t)\) such that \[
    \text{cosh}(W(T))=z+\int_0^Th(t)\ dW(t).
    \]
  \end{enumerate}
\end{enumerate}

Let \(L(t)\) be a Likelihood process and let \(dQ=L(T)dP\) be a new
probability measure.

\begin{enumerate}
\def\labelenumi{\alph{enumi}.}
\setcounter{enumi}{2}
\tightlist
\item
  Determine the Likelihood process \(L(t)\) such that
  \(\text{sinh}(W(t))\) is a martingale under the probability measure
  \(Q\).
\end{enumerate}

\textbf{Solution (a).}

Assume that \(X(t)=\text{sinh}(W(t))\) and \(Y(t)=\text{cosh}(W(t))\).
The relevant derivatives is then

\[
\frac{d}{dw}\text{sinh}(w)=\frac{1}{2}e^w+\frac{1}{2}e^{-w}=\text{cosh}(w),\ \frac{d^2}{dw^2}\text{sinh}(w)=\frac{d}{dw}\text{cosh}(w)=\frac{1}{2}e^w-\frac{1}{2}e^{-w}=\text{sinh}(w).
\]

That is sinus and cosinus hyperbolic are their each others derivative.
Then by Ito's formula we have

\begin{align*}
dX(t)&=\text{cosh}(W(t))\ dW(t)+\frac{1}{2}\text{sinh}(W(t))\ (dW(t))^2\\
&=\frac{1}{2}\text{sinh}(W(t))\ dt+\text{cosh}(W(t))\ dW(t)\\
&=\frac{1}{2}X(t)\ dt+Y(t)\ dW(t).
\end{align*}

and

\begin{align*}
dY(t)&=\text{sinh}(W(t))\ dW(t)+\frac{1}{2}\text{cosh}(W(t))\ (dW(t))^2\\
&=\frac{1}{2}\text{cosh}(W(t))\ dt+\text{sinh}(W(t))\ dW(t)\\
&=\frac{1}{2}Y(t)\ dt+X(t)\ dW(t).
\end{align*}

And thus the result has been prooved. \(\square\)

\textbf{Solution (b).}

\emph{(i)}: Consider the function \(f(z,y)=zy\). Then we have that
\(M(t)=f(Z(t),Y(t))\) for \(Z(t)=e^{-t/2}\) hence \(M\) has dynamics
given by Ito's formula:

\begin{align*}
dM(t)&=df(Z(t),Y(t))\\
&=Y(t)\ dZ(t)+Z(t)\ dY(t)+(dZ(t))(dY(t))\\
&=Y(t)\ (-\frac{1}{2}Z(t)\ dt)+Z(t)\ (\frac{1}{2}Y(t)\ dt+X(t)\ dW(t))+(-\frac{1}{2}Z(t)\ dt)(\frac{1}{2}Y(t)\ dt+X(t)\ dW(t))\\
&=X(t)Z(t)\ dW(t).
\end{align*}

and so we see that pr. lemma 4.11 \(M\) is a martingale. \(\square\)

\emph{(ii)}: We have from above

\begin{align*}
M(T)&=M(0)+\int_0^TX(t)Z(t)\ dW(t)=Z(T)\text{cosh}(W(t))
\end{align*}

Hence it follows that

\[
\text{cosh}(W(T))=\frac{M(0)}{Z(T)}+\int_0^T\frac{X(t)Z(t)}{Z(T)}\ dW(t).
\]

Using that the martingale has initial value

\[
M(0)=e^{-0/2}\text{cosh}(0)=1
\]

we have

\[
\text{cosh}(W(T))=e^{T/2}+\int_0^T \text{sinh}(W(t))e^{(T-t)/2}\ dW(t).
\]

In total we have \(z=e^{T/2}\) and \(h(t)=\text{sinh}(W(t))e^{(T-t)/2}\)
as desired. \(\square\)

\textbf{Solution (c).}

We have that under the measure \(Q\) the dynamics of \(W\) is given by
the Girsanov Theorem

\[
dW(t)=\varphi\ dt+dW^Q_t,
\]

where \(\varphi\) is the Girsanov kernel associated with \(L\). Then we
know that \(X\) has dynamics under the \(Q\) measure:

\begin{align*}
dX(t)&=\frac{1}{2}X(t)\ dt+Y(t)\ (\varphi\ dt+dW^Q_t)\\
&=\left(\frac{1}{2}X(t)+\varphi Y(t)\right)\ dt+Y(t)\ dW^Q_t
\end{align*}

and so we would have that \(X\) is a martingale under \(Q\) if

\[
\varphi_t=-\frac{1}{2}\frac{X(t)}{Y(t)}=-\frac{1}{2}\frac{\text{sinh}(W(t))}{\text{cosh}(W(t))}=-\frac{1}{2}\text{tanh}(W(t)).
\]

Then we can define a Likelihood process with initial condition \(L_0=1\)
and dynamics \(dL_t=\varphi_tL_t\ dW(t)\) i.e.~\(L\) is the solution

\[
L_t=\exp\left\{\int_0^s-\frac{1}{2}\text{tanh}(W(s)) \ dW(t)-\frac{1}{2}\int_0^t\left(-\frac{1}{2}\text{tanh}(W(t))\right)^2 \ dW(t)\right\}> 0.
\]

We lastly show that the Novikov condition is satisfied i.e.

\begin{align*}
E^P\left[e^{\frac{1}{2}\int_0^T\Vert \varphi_s\Vert^2\ ds}\right]&=E^P\left[e^{\frac{1}{8}\int_0^T\text{tanh}^2(W(t))\ ds}\right]<\infty
\end{align*}

and so \(L\) is a \(P\)-martingale and \(L\) is a Likelihood process. We
thus have found a Likelihood process such that \(X\) is a martingale
under the measure \(Q\) given by \(dQ=L_T\ dP\). \(\square\)

\hypertarget{exam-202021}{%
\subsection{Exam 2020/21}\label{exam-202021}}

Some content

\hypertarget{exam-202122}{%
\subsection{Exam 2021/22}\label{exam-202122}}

Some content

\hypertarget{exam-202223}{%
\subsection{Exam 2022/23}\label{exam-202223}}

Some content

\end{document}